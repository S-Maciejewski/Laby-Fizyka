\documentclass[10pt,a4paper]{article}

\usepackage{polski}
\usepackage[utf8]{inputenc}

\usepackage[polish]{babel}
\usepackage{hhline}
\usepackage{pgfplots}

\usepackage{geometry}
\geometry{a4paper, total={170mm,257mm}, left=20mm, top=20mm }

\author{Sebastian Maciejewski 132275 i Jan Techner 132332}
\title{Wyznaczanie zależności przewodnictwa od temperatury 
dla półprzewodników i przewodników - \\ doświadczenie 203 (sala 217A)}
\date{8 grudnia 2017}
\setlength{\parindent}{0pt}
\newcommand{\forceindent}{\leavevmode{\parindent=3em\indent}}
\begin{document}
\maketitle
\section{Wstęp teorytyczny}
\forceindent Przewodnictwo właściwe materiałów zależy od temperatury. Dla metali (przewodników) spada przy wzroście temperatury ze względu na spadek ruchliwości nośników. W przypadku półprzewodnika samoistnego zdolność przewodzenia prądu rośnie wykładniczo przy wzroście temperatury. Dzieje się tak, gdyż rośnie koncentracja nośników. Ruchliwość spada podobnie jak w metalach, zmiany te są jednak niewielkie w porównaniu ze zmianami koncentracji.\\
Takie właściwości półprzewodnika wynikają z tego, iż nośnikami prądu są w nim elektrony w paśmie przewodnictwa i dziury w paśmie walencyjnym. Elektrony są dostarczane do pasma przewodnictwa z pasma walencyjnego (w półprzewodnikach samoistnych) lub z poziomów domieszkowych (w półprzewodnikach domieszkowanych). Dziury natomiast powstają w paśmie walencyjnym po przejściu elektronu do pasma przewodnictwa.\\
Liczba elektronów przechodzących na wyższy poziom energetyczny zależy wykładniczo min. od temperatury i wyraża się (dla półprzewodników samoistnych) wzorem:
\begin{equation}
n = n_{0_s} e^{\frac{E_g}{2kT}}
\end{equation}
gdzie $E_g$ to szerokość pasma zabronionego, $k$ to stała Boltzmana a $T$ temperatura.

\subsection*{Opis doświadczenia}

\section{Wyniki pomiarów}
\forceindent Dla ogrzewania i chłodzenia przewodnika i półprzewodnika otrzymaliśmy następujące odczyty oporu:\\
\begin{center}
\begin{tabular}{|c|c|c|}
\hline
Temperatura $(^{\circ}C)$ & Opór półprzewodnika $(k\Omega)$ & Opór przewodnika $(\Omega)$\\
\hline
22,8&208,0&109,1\\ 
 \hline 
26,3&177,0&110,4\\ 
 \hline 
31,3&144,0&112,1\\ 
 \hline 
36,3&117,0&114,1\\ 
 \hline 
41,3&95,0&116,0\\ 
 \hline 
46,3&80,1&117,9\\ 
 \hline 
51,3&66,5&119,8\\ 
 \hline 
56,3&54,7&121,7\\ 
 \hline 
61,3&46,4&123,5\\ 
 \hline 
66,3&39,2&125,3\\ 
 \hline 
71,3&33,0&127,1\\ 
 \hline 
76,3&28,0&129,1\\ 
 \hline 
81,3&23,9&130,8\\ 
 \hline 
86,3&20,4&132,6\\ 
 \hline 
81,3&26,8&131,3\\ 
 \hline 
76,3&31,6&129,9\\ 
 \hline 
71,3&38,1&128,1\\ 
 \hline 
66,3&44,9&126,4\\ 
 \hline 
61,3&52,7&124,6\\ 
 \hline 
56,3&61,2&122,9\\ 
 \hline 
51,3&73,4&120,7\\ 
 \hline 
46,3&87,0&118,6\\ 
 \hline 
41,3&103,6&116,6\\ 
 \hline 
36,3&124,7&114,6\\ 
 \hline 
31,3&149,9&112,5\\ 
 \hline 
26,3&181,6&110,5\\ 
 \hline 
24,9&191,7&109,9\\ 
 \hline 



\end{tabular}
\end{center}



\section{Opracowanie wyników}
\forceindent Dla zależności:
\begin{equation}
ln(1/R) = f(1/T)
\end{equation}  
wyliczymy teraz, korzystając z metody regresji liniowej, współczynnik nachylenia prostej.\\
Przyjmujemy, że $ln(1/R) = y$ i $1/T = x$.
Posługując się metodą regresji liniowej opisaną wzorem:
\begin{equation}
a=\frac{n\Sigma x_i y_i - \Sigma x_i \Sigma y_i}{n\Sigma x_i^2 - (\Sigma x_i)^2},
\end{equation}
wyznaczamy współczynnik nachylenia prostej $a$, oraz jego niepewność. 
\begin{equation}
a = -3869,397 \left[\frac{K}{\Omega}\right]
\end{equation}
Następnie korzystając z równania:
\begin{equation}
a = \frac{E_A}{2k} \Rightarrow E_A = 2ak
\end{equation}
obliczamy energię aktywacji ($E_A$), która wynosi:
$$ E_A = -1,068 * 10^-19 \frac{J}{K} = -0,667 \frac{eV}{K} $$

Błąd wyznaczenia wielkości $a$:\\
$$ \Delta a = \sqrt{\frac{n(\Sigma y_i ^2 - a \Sigma x_i y_i - b\Sigma y_i)}{(n-2)(n \Sigma x_i ^2 - (\Sigma x_i)^2)}} = $$


\forceindent Zatem ostateczne wartości $a$ i $E_A$ wyglądają następująco:

\begin{table}[!h]
\centering
\begin{tabular}{|cc||c|c|}
\multicolumn{1}{c}{} & \multicolumn{1}{c}{$a$} & \multicolumn{1}{c}{$E_A [\frac{J}{K}]$} & \multicolumn{1}{c}{$E_A [\frac{eV}{K}]$}\\
\hline
pomiar & $-3869,39702854943$ & $-1,068 * 10^-19$ & $-0,667$\\
\hline
dokładność & TODO & TODO & TODO\\
\hline
po zaokrągleniu & TODO  & TODO & TODO\\
\hline
\end{tabular}
\caption{Współczynnik nachylenia linii $a$ i energia aktywacji $E_A$ wraz z dokładnościami $\Delta a$ i $\Delta E_A$}
\end{table}

\forceindent Wartości temperatury w $K$, obliczenia $1/T$ i $ln(1/R)$.

\begin{center}
\begin{tabular}{|c|c|c|}
\multicolumn{1}{c}{$T (K)$} & \multicolumn{1}{c}{$1/T$} & \multicolumn{1}{c}{$ln(1/R)$ dla półprzewodnika}\\
\hline
295,95&0,00337895&-12,24529336\\ 
 \hline 
299,45&0,00333946&-12,08390501\\ 
 \hline 
304,45&0,00328461&-11,87756858\\ 
 \hline 
309,45&0,00323154&-11,66992921\\ 
 \hline 
314,45&0,00318016&-11,46163217\\ 
 \hline 
319,45&0,00313038&-11,29103113\\ 
 \hline 
324,45&0,00308214&-11,10495723\\ 
 \hline 
329,45&0,00303536&-10,90961899\\ 
 \hline 
334,45&0,00298998&-10,74505474\\ 
 \hline 
339,45&0,00294594&-10,57643203\\ 
 \hline 
344,45&0,00290318&-10,40426284\\ 
 \hline 
349,45&0,00286164&-10,23995979\\ 
 \hline 
354,45&0,00282127&-10,08163374\\ 
 \hline 
359,45&0,00278203&-9,92329018\\ 
 \hline 
354,45&0,00282127&-10,19615717\\ 
 \hline 
349,45&0,00286164&-10,3609124\\ 
 \hline 
344,45&0,00290318&-10,54796956\\ 
 \hline 
339,45&0,00294594&-10,71219307\\ 
 \hline 
334,45&0,00298998&-10,87237073\\ 
 \hline 
329,45&0,00303536&-11,02190247\\ 
 \hline 
324,45&0,00308214&-11,20367921\\ 
 \hline 
319,45&0,00313038&-11,3736634\\ 
 \hline 
314,45&0,00318016&-11,54829261\\ 
 \hline 
309,45&0,00323154&-11,73366613\\ 
 \hline 
304,45&0,00328461&-11,91772368\\ 
 \hline 
299,45&0,00333946&-12,10956175\\ 
 \hline 
298,05&0,00335514&-12,16368693\\ 
 \hline 

\end{tabular}
\end{center}



\end{document}