\documentclass[10pt,a4paper]{article}

\usepackage{polski}
\usepackage[utf8]{inputenc}

\usepackage[polish]{babel}
\usepackage{hhline}
\usepackage{pgfplots}

\usepackage{geometry}
\geometry{a4paper, total={170mm,257mm}, left=20mm, top=20mm }

\author{Sebastian Maciejewski 132275 i Jan Techner 132332}
\title{Badanie momentu bezwładności - \\ doświadczenie 104 (sala 217)}
\date{24 listopada 2017}
\setlength{\parindent}{0pt}
\newcommand{\forceindent}{\leavevmode{\parindent=3em\indent}}
\begin{document}
\maketitle
\section{Wstęp teorytyczny}
\forceindent W opisie dynamiki ruchu postępowego pojawia się pojęcie bezwładności związane z masą \textit{m} poruszającego się ciała. W przypadku ruchu obrotowego znajomość masy ciała jest niewystarczająca, istotny
jest również jej przestrzenny rozkład względem osi obrotu. Wielkością fizyczną zawierającą informacje o
masie ciała i jej przestrzennym rozkładzie względem osi obrotu jest moment bezwładności \textit{I}.\\
\forceindent Dla pojedynczego punktu materialnego o masie m wirującego wokół osi oddalonej od niego o
odległość \textit{r} możemy zapisać następującą zależność na moment bezwładności:
\begin{equation}
I=mr^2
\end{equation}
W przypadku układu N punktów materialnych sztywno połączonych ze sobą względem osi obrotu, zwanej osią bezwładności, moment bezwładności układu jest równy sumie momentów bezwładności poszczególnych punktów materialnych.
\begin{equation}
I=m_1r_{1}^{2} + m_2r_{2}^{2} + ... + m_Nr_{N}^{2} = \sum\limits_{i=1}^{N} m_{i}r_{i}^{2}.
\end{equation}
gdzie $m_{i}$ jest jest masą $i$-tego punktu materialnego, a $r_{i}$ jego odległością od osi bezwładności.
\subsection*{Twierdzenie Steinera} 
\forceindent Jeżeli chcemy obliczyć moment bezwładności względem dowolnej osi nieprzechodzącej przez środek masy bryły, przydatna staje się twierdznie Steinera. Mówi ono, że jeżeli moment bezwładności bryły sztywnej względem osi przechodzącej przez jej środek masy równa się $I_{0}$ to moment bezwładności tej bryły obracającej się względem innej osi równoległej do osi przechodzącej przez jej środek masy wynosi:
\begin{equation}
I = I_{0} + md^2,
\end{equation}
gdzie $m$ jest masą bryły, a $d$ odległością między osiami.

\subsection*{Opis doświadczenia}

\forceindent W ćwiczeniu zostaną wyznaczone momenty bezwładności stalowego pręta oraz dysku. Dodatkowym
zadaniem będzie eksperymentalne potwierdzenie twierdzenia Steinera. Do badań posłuży wahadło skrętne
złożone ze stabilnej podstawy oraz pionowej osi osadzonej na łożyskach o bardzo małym tarciu. Oś oraz
podstawa połączone są przy pomocy spiralnej sprężyny, która umożliwia wahania skrętne. Na końcu osi
znajduje się śruba umożliwiająca mocowanie na niej brył.\\
\forceindent Wahadło skrętne jest szczególnym przypadkiem wahadła fizycznego. Jeżeli założymy, że wychylenia
wahadła są niewielkie (do około $180^{\circ}$) oraz zaniedbamy siły oporu, jego ruch można opisać jako ruch
harmoniczny prosty. W takim przypadku okres T drgań wahadła można zapisać następująco:
\begin{equation}
T = 2 \pi \sqrt{\frac{I}{D}},
\end{equation}
gdzie $I$ jest momentem bezwładności bryły zamocowanej na osi wahadła, a $D$ jest parametrem charakterystycznym dla danej sprężyny - jej momentem kierującym.

\newpage
\section{Wyniki pomiarów}
\forceindent Masa ciężarków, masa i długość pręta, odległości między nacięciami na pręcie zostały zmierzone na początku doświadczenia i wynosiły:\\
\begin{center}
\begin{tabular}{|c|c|c|}
\hline
Mierzona wartość & Pomiar & Dokładność pomiarowa\\
\hhline{|=|=|=|}
Masa ciężarków & 0.259 [kg] & $\pm$ 0.001 [kg]\\
Masa pręta & 0.136 [kg] & $\pm$ 0.001 [kg]\\
Masa dysku & 0.435 [kg] & $\pm$ 0.001 [kg]\\
\hline
Długość pręta & 0.62 [m] & $\pm$ 0.001 [m]\\
Średnica dysku & 0.32 [m] & $\pm$ 0.001 [m]\\
\hline
Odległość między nacięciami & 0.05 [m] & $\pm$ 0.001 [m]\\
\hline

\end{tabular}
\end{center}
\vspace{10pt}

\forceindent Dokładność pomiaru czasu ze względu na pomiar manualny wynosi $\pm$ 0.1s
\vspace{10pt}
\begin{table}[!h]
\centering
\begin{tabular}{|c||c|c|c||c|c|}
\hline
Odległość od osi obrotu [cm]& Próba 1 [s] & Próba 2 [s] & Próba 3 [s] & Śr. 5T [s] & Śr. T [s]\\
\hhline{|=||=|=|=||=|=|}
0 & 12.45 & 12.50 & 12.60 & 12.517 & 2.503\\
5 & 14.21 & 14.16 & 14.37 & 14.247 & 2.849\\ 
10 & 18.39 & 18.46 & 18.62 & 18.490 & 3.698\\
15 & 24.04 & 24.09 & 24.00 & 24.043 & 4.809\\
20 & 30.20 & 30.06 & 29.75 & 30.003 & 6.001\\
25 & 36.29 & 36.47 & 36.53 & 36.430 & 7.286\\
30 & 42.87 & 42.83 & 42.91 & 42.870 & 8.574\\
\hline

\end{tabular}
\caption{Pomiary czasu 5 wahnięć pręta przy różnych odległościach ciężarków od osi obrotu}
\end{table}
\vspace{10pt}


\begin{table}[!h]
\centering
\begin{tabular}{|c||c|c|c||c|c|}
\hline
Odległość od osi obrotu [cm]& Próba 1 [s] & Próba 2 [s] & Próba 3 [s] & Śr. 5T [s] & Śr. T [s]\\
\hhline{|=||=|=|=||=|=|}
0 & 14.14 & 14.18 & 14.23 & 14.185 & 2.837\\
2 & 14.26 & 14.22 & 14.18 & 14.22 & 2.844\\ 
4 & 14.59 & 14.41 & 14.58 & 14.525 & 2.905\\
6 & 15.77 & 15.86 & 15.98 & 15.87 & 3.174\\
8 & 17.27 & 17.14 & 17.27 & 17.225 & 3.445\\
10 & 19.08 & 18.97 & 18.99 & 19.015 & 3.803\\
12 & 20.96 & 20.81 & 20.76 & 20.845 & 4.169\\
14 & 22.80 & 22.84 & 22.83 & 22.825 & 4.565\\
\hline

\end{tabular}
\caption{Pomiary czasu 5 wahnięć dysku przy różnych odległościach od osi obrotu}
\end{table}
\vspace{10pt}

\section{Opracowanie wyników}

\begin{figure}[!h]
\centering
\begin{tikzpicture}[scale=1]
\begin{axis}[
xlabel={Kwadrat odległości ciężarków od osi obrotu $r^2$},
ylabel={Kwadrat okresu wahnięcia $T^2$},
xmin=0,xmax=0.11,
ymin=0,ymax=80,
legend pos=north west,
ymajorgrids=true,grid style=dashed
]

%mosiądz - rozszerzanie
\addplot[smooth, tension={0.3}, green, mark=*, mark size = {0.5pt}]
coordinates {
(0, 6.265009)
(0.0025, 8.116801)
(0.01, 13.675204)
(0.0225, 23.126481)
(0.04, 36.012001)
(0.0625, 53.085796)
(0.09, 73.513476)
};

\draw [draw=black] (0.5,79.17) rectangle ++(4, 4);
\draw [draw=black] (8,134.75) rectangle ++(4, 4);
\draw [draw=black] (20.5,229.26) rectangle ++(4, 4);
\draw [draw=black] (38,358.12) rectangle ++(4, 4);
\draw [draw=black] (60.5,528.86) rectangle ++(4, 4);
\draw [draw=black] (88,733.13) rectangle ++(4, 4);

\legend{$f(r^2)$}
\end{axis}
\end{tikzpicture}
\caption{Zależność wydłużenia pręta od zmiany temperatury podczas ogrzewania}
\label{fig:wyk}
\end{figure}

\forceindent W celu obliczenia momentów bezwładności z danych pomiarowych posłużymy się równaniem :
\begin{equation}
T^2 = \frac{8 \pi^2 m_c}{D}r^2 + T_p^2
\end{equation}
gdzie $T$ jest okresem drgań naszego wahadła, $m_c$ jest zmierzoną masą ciężarka, $r$ jest odległością ciężarków (lub środka dysku) od osi obrotu, zaś $D$ jest momentem kierującym.\\
\forceindent Równanie $(3)$ oznacza, że zależność kwadratu okresów $T$ od kwadratów odległości $r$ jest liniowa (co w przybliżeniu widać na wykresie). Można zatem policzyć współczynnik nachylenia prostej przy pomocy metody regresji liniowej (gdzie $x = r^2$ i $y = T^2$). Współczynnik $a$ wyraża się wzorem:

\begin{equation}
a=\frac{n\Sigma x_i y_i - \Sigma x_i \Sigma y_i}{n\Sigma x_i^2 - (\Sigma x_i)^2},
\end{equation}

zaś jego jednostką jest $ a=\frac{\frac{1}{s^2}*m^2-\frac{1}{s^2}*m^2}{m^4-m^4}=\frac{s^2}{m^2} $.\\

W takiej sytuacji możemy łatwo policzyć $D$ ze wzoru:
\begin{equation}
D = \frac{8\pi ^2 m_c}{a},
\end{equation}
jednostką $D$ jest, jak widać $ D=\frac{kg}{\frac{s^2}{m^2}}=\frac{kg*m^2}{s^2} $.

\begin{table}[!h]
\centering
\begin{tabular}{|cc||c|}
\multicolumn{1}{c}{} & \multicolumn{1}{c}{$a$} & \multicolumn{1}{c}{$D$}\\
\hline
pomiar & 24366,1434802235$\pm$& 0,000839271932206056$\pm$\\
\hline
dokładność & zmyślona & zmyślona\\
\hline
po zaokrągleniu & ileś tam  & ileś tam \\
\hline
\end{tabular}
\caption{Współczynnik nachylenia linii $a$ i moment kierujący $D$ wraz z dokładnościami $\Delta a$ i $\Delta D$}
\end{table}

\forceindent Z wyników uzyskanych dla nieobciążonego pręta obliczyliśmy jego moment bezwładności wraz z niepewnością jako:
$ I = \frac{0,136kg * (0,62m)^2}{12} = (0,00436 \pm 0,00300) kg*m^2 $

\forceindent W podobny sposób, znając średni okres $T = 2,837 s$ obrotu dysku wokół osi przechodzącej przez jego środek, możemy policzyć jego moment bezwładności, który wynosi
$ I = \frac{0,435kg*(0,16m)^2}{2} = (0,00557 \pm 0,00300) kg*m^2 $

\begin{center}
\begin{tabular}{|c|c|c|c|c|}
\hline	
 & Śr. Okres T & Teoretyczny & Zmierzony & Niepewność\\
\hline
Pręt & 2,503 & 0,004357 & 0,004212 & 0,003000\\
\hline
Ciężarki 5 cm od osi & 2,849&0,005652&	0,005457&	0,006000\\
\hline
Ciężarki 10 cm od osi&3,698	&0,009537&	0,009192&	0,006000\\
\hline
Ciężarki 15 cm od osi&4,809 &0,016012&	0,015542&	0,006000\\
\hline
Ciężarki 20 cm od osi&6,001 &0,025077&	0,024203&	0,006000\\
\hline
Ciężarki 25 cm od osi&7,286 &0,036732&	0,035682&	0,006000\\
\hline
Ciężarki 30 cm od osi&8,574	&0,050977&	0,049412&	0,006000\\
\hline
Dysk&2,837&0,005568&	0,005409&	0,003000\\
\hline
Dysk 2 cm od osi&2,844&0,005742&	0,005437&	0,006000\\
\hline
Dysk 4 cm od osi&2,905&0,006264&	0,005674&	0,006000\\
\hline
Dysk 6 cm od osi&3,174&0,007134&	0,006771&	0,006000\\
\hline
Dysk 8 cm od osi&3,445&0,008352&	0,007979&	0,006000\\
\hline
Dysk 10 cm od osi&3,803&0,009918&	0,009719&	0,006000\\
\hline
Dysk 12 cm od osi&4,169&0,011832&	0,011680&	0,006000\\
\hline
Dysk 14 cm od osi&4,565&0,014094&	0,014005&	0,006000\\
\hline


\end{tabular}
\end{center}


\end{document}