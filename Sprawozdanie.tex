\documentclass[10pt,a4paper]{article}

\usepackage{polski}
\usepackage[utf8]{inputenc}

\usepackage[polish]{babel}
\usepackage{hhline}
\usepackage{pgfplots}

%\usepackage{slashbox}
\usepackage{graphicx}
\usepackage{caption}
\usepackage{subcaption}
\usepackage{colortbl}

\usepackage{geometry}
\geometry{a4paper, total={170mm,257mm}, left=20mm, top=20mm }

\author{Sebastian Maciejewski 132275 i Jan Techner 132332}
\title{Badanie oddziaływania pola magnetycznego na przewodnik z prądem - \\ doświadczenie 204 (sala 217A)}
\date{22 grudnia 2017}
\setlength{\parindent}{0pt}
\newcommand{\forceindent}{\leavevmode{\parindent=3em\indent}}
\begin{document}
\maketitle
\section{Wstęp teoretyczny}
\forceindent Oddziaływanie pola magnetycznego na przewodnik z prądem jest zjawiskiem powszechnie wykorzystywanym w technice. W licznych doświadczeniach z cząsteczkami obdarzonymi ładunkiem elektrycznym poruszającymi się w polu magnetycznym zaobserwowano występowanie siły powodującej zakrzywienie ich toru. Siłę tę jako pierwszy opisał Holenderski fizyk Hendrik Lorentz, a konsekwencją występowania siły Lorentza jest siła działająca w polu magnetycznym na przewodnik z prądem. Siła ta bierze się z tego, że w odcinku o długości $l$ w danej chwili przepływa $n$ elektronów o ładunku $e$ i średniej prędkości unoszenia $v_u$ i na każdy elektron działa siła Lorentza.

\subsection*{Opis doświadczenia}
\forceindent Doświadczenie polega na zbadaniu siły elektrodynamicznej działającej na ramkę w polu magnetycznym. Chcąc wyznaczyć siłę elektrodynamiczną działającą na dolny bok ramki, musimy rozważyć momenty sił występujące po wychyleniu ramki z położenia równowagi. Momenty te będą związane z siłami: grawitacji oraz elektrodynamiczną. \\
Szukana przez nas siła będzie miała postać 
\begin{equation}
F_ED = \frac{1}{2} tg(\alpha)mg,
\end{equation}
gdzie $m$ to masa ramki i $\alpha$ to kąt odchylenia.\\
Ponieważ widać, że $tg(\alpha)$ będzie równy stosunkowi długości odcinka $x$ odczytanej ze skali do odległości osi obrotu od środka skali $d$, można powyższy wzór przedstawić jako
\begin{equation}
F_{ED} = \frac{mgx}{2d},
\end{equation}
a z racji tego, że wielkości $m$, $g$ i $d$ są stałe finalna postać wzoru (po podstawieniu $c=\frac{mg}{2d}$, gdzie stała $c =  2,65 \pm 0,05 \frac{N}{m}$), to :
\begin{equation}
F_{ED} = cx
\end{equation}


\newpage
\section{Wyniki pomiarów}

Wyniki pomiarów natężenia prądu $I[A]$ w zależności od wychylenia i liczby zwojów\\

\begin{table}[!h]
\centering
\begin{tabular}{|>{\columncolor[gray]{0.8}}c|c|c|c|c|c|c|c|c|c|c|}
%\multicolumn{1}{c}{zwoje} & \multicolumn{10}{c}{wychylenie w $m$}
\hline
%\backslashbox{Zwoje}{Wychylenie [$m$]}
\rowcolor{lightgray} &0,01&0,02&0,03&0,04&0,05&0,06&0,07&0,08&0,09&0,1\\ \hline
 
5&0,128&0,281&0,421&0,566&0,706&0,859&1,008&1,118&1,281&1,426\\ \hline
10&0,065&0,141&0,208&0,277&0,351&0,432&0,501&0,592&0,652&0,732\\ \hline
15&0,046&0,091&0,143&0,195&0,242&0,282&0,337&0,39&0,444&0,495\\ \hline
20&0,037&0,075&0,102&0,143&0,186&0,213&0,261&0,298&0,341&0,381\\ \hline
25&0,02&0,049&0,088&0,118&0,144&0,175&0,202&0,241&0,275&0,308\\ \hline

\end{tabular}
\caption{Wychylenie w prawo}
\end{table}

\begin{table}[!h]
\centering
\begin{tabular}{|>{\columncolor[gray]{0.8}}c|c|c|c|c|c|c|c|c|c|c|}
%\multicolumn{1}{c}{zwoje} & \multicolumn{10}{c}{wychylenie w $m$}
\hline
%\backslashbox{Zwoje}{Wychylenie [$m$]}
\rowcolor{lightgray} &-0,01&-0,02&-0,03&-0,04&-0,05&-0,06&-0,07&-0,08&-0,09&-0,1\\  \hline 

5&0,145&0,289&0,426&0,596&0,738&0,882&1,057&1,213&1,381&1,557\\ \hline
10&0,075&0,141&0,218&0,301&0,374&0,454&0,529&0,616&0,694&0,781\\ \hline
15&0,047&0,098&0,139&0,198&0,245&0,307&0,354&0,417&0,478&0,525\\ \hline
20&0,036&0,071&0,112&0,152&0,187&0,227&0,278&0,315&0,369&0,419\\ \hline
25&0,027&0,055&0,085&0,122&0,147&0,186&0,218&0,248&0,293&0,323\\ \hline

\end{tabular}
\caption{Wychylenie w lewo}
\end{table}
\section{Opracowanie wyników}

\forceindent Korzystając ze wzoru $(3)$ możemy obliczyć siłę elektrodynamiczną działającą na pręt w zależności od wielkości wychylenia z położenia równowagi.

\begin{table}[!h]
\centering
\begin{tabular}{|c|c|}
\hline
Wychylenie ramki (m) & Wartość siły $F_{ED}$ [N]\\
\hline
0,01&0,0265\\ 
 \hline 
0,02&0,053\\ 
 \hline 
0,03&0,0795\\ 
 \hline 
0,04&0,106\\ 
 \hline 
0,05&0,1325\\ 
 \hline 
0,06&0,159\\ 
 \hline 
0,07&0,1855\\ 
 \hline 
0,08&0,212\\ 
 \hline 
0,09&0,2385\\ 
 \hline 
0,1&0,265\\ 
 \hline 
\end{tabular}
\caption{Wartość siły $F_{ED}$ w zależności od wychylenia}
\end{table}
\newpage

\forceindent Na poniższych wykresach przedstawiono zależności siły elektrodynamicznej $F_{ED}$ od natężenia prądu przepływającego przez uzwojenia ramki dla wszystkich wartości liczby uzwojeń $n$ :


\begin{figure}[!h]
\centering
\begin{minipage}{.5\textwidth}
\centering
\caption*{Wychylenie w lewo}
\begin{tikzpicture}[scale=0.9]
\begin{axis}[
xlabel={Natężenie prądu $I$ [A]},
ylabel={Wartość siły $F_{ED}$ [N]},
xmin=0,xmax=1.7,
ymin=0,ymax=0.3,
legend pos=south east,
ymajorgrids=true,
xmajorgrids=true,
grid style=dashed]

\addplot[smooth, tension={0.3}, red, mark=*, mark size = {0.5pt}]
coordinates {
(0.145, 0.0265)
(0.289, 0.053)
(0.426, 0.0795)
(0.596, 0.106)
(0.738, 0.1325)
(0.882, 0.159)
(1.057, 0.1855)
(1.213, 0.212)
(1.381, 0.2385)
(1.557, 0.265)
};

\addplot[smooth, tension={0.3}, orange, mark=*, mark size = {0.5pt}]
coordinates {
(0.075, 0.0265)
(0.141, 0.053)
(0.218, 0.0795)
(0.301, 0.106)
(0.374, 0.1325)
(0.454, 0.159)
(0.529, 0.1855)
(0.616, 0.212)
(0.694, 0.2385)
(0.781, 0.265)
};

\addplot[smooth, tension={0.3}, green, mark=*, mark size = {0.5pt}]
coordinates {
(0.047, 0.0265)
(0.098, 0.053)
(0.139, 0.0795)
(0.198, 0.106)
(0.245, 0.1325)
(0.307, 0.159)
(0.354, 0.1855)
(0.417, 0.212)
(0.478, 0.2385)
(0.525, 0.265)
};

\addplot[smooth, tension={0.3}, cyan, mark=*, mark size = {0.5pt}]
coordinates {
(0.036, 0.0265)
(0.071, 0.053)
(0.112, 0.0795)
(0.152, 0.106)
(0.187, 0.1325)
(0.227, 0.159)
(0.278, 0.1855)
(0.315, 0.212)
(0.369, 0.2385)
(0.419, 0.265)
};

\addplot[smooth, tension={0.3}, blue, mark=*, mark size = {0.5pt}]
coordinates {
(0.027, 0.0265)
(0.055, 0.053)
(0.085, 0.0795)
(0.122, 0.106)
(0.147, 0.1325)
(0.186, 0.159)
(0.218, 0.1855)
(0.248, 0.212)
(0.293, 0.2385)
(0.323, 0.265)
};

\legend{5 zwojów, 10 zwojów, 15 zwojów, 20 zwojów, 25 zwojów}
\end{axis}
\end{tikzpicture}
\end{minipage}%
\begin{minipage}{.5\textwidth}
\centering
\caption*{Wychylenie w prawo}
\begin{tikzpicture}[scale=0.9]
\begin{axis}[
xlabel={Natężenie prądu $I$ [A]},
ylabel={Wartość siły $F_{ED}$ [N]},
xmin=0,xmax=1.7,
ymin=0,ymax=0.3,
legend pos=south east, 
ymajorgrids=true,
xmajorgrids=true,
grid style=dashed]

\addplot[smooth, tension={0.3}, red, mark=*, mark size = {0.5pt}]
coordinates {
(0.128, 0.0265)
(0.281, 0.053)
(0.421, 0.0795)
(0.566, 0.106)
(0.706, 0.1325)
(0.859, 0.159)
(1.008, 0.1855)
(1.118, 0.212)
(1.281, 0.2385)
(1.426, 0.265)
};

\addplot[smooth, tension={0.3}, orange, mark=*, mark size = {0.5pt}]
coordinates {
(0.065, 0.0265)
(0.141, 0.053)
(0.208, 0.0795)
(0.277, 0.106)
(0.351, 0.1325)
(0.432, 0.159)
(0.501, 0.1855)
(0.592, 0.212)
(0.652, 0.2385)
(0.732, 0.265)
};

\addplot[smooth, tension={0.3}, green, mark=*, mark size = {0.5pt}]
coordinates {
(0.046, 0.0265)
(0.091, 0.053)
(0.143, 0.0795)
(0.195, 0.106)
(0.242, 0.1325)
(0.282, 0.159)
(0.337, 0.1855)
(0.39, 0.212)
(0.444, 0.2385)
(0.495, 0.265)
};

\addplot[smooth, tension={0.3}, cyan, mark=*, mark size = {0.5pt}]
coordinates {
(0.037, 0.0265)
(0.075, 0.053)
(0.102, 0.0795)
(0.143, 0.106)
(0.186, 0.1325)
(0.213, 0.159)
(0.261, 0.1855)
(0.298, 0.212)
(0.341, 0.2385)
(0.381, 0.265)
};

\addplot[smooth, tension={0.3}, blue, mark=*, mark size = {0.5pt}]
coordinates {
(0.02, 0.0265)
(0.049, 0.053)
(0.088, 0.0795)
(0.118, 0.106)
(0.144, 0.1325)
(0.175, 0.159)
(0.202, 0.1855)
(0.241, 0.212)
(0.275, 0.2385)
(0.308, 0.265)

};
%\draw [draw=black] (0.5,79.17) rectangle ++(4, 4);
%\draw [draw=black] (8,134.75) rectangle ++(4, 4);
%\draw [draw=black] (20.5,229.26) rectangle ++(4, 4);
%\draw [draw=black] (38,358.12) rectangle ++(4, 4);
%\draw [draw=black] (60.5,528.86) rectangle ++(4, 4);
%\draw [draw=black] (88,733.13) rectangle ++(4, 4);
\legend{5 zwojów, 10 zwojów, 15 zwojów, 20 zwojów, 25 zwojów}
\end{axis}
\end{tikzpicture}
\end{minipage}
\caption{Zależność wartości siły $F_{ed}$ od natężenia prądu $I$ dla wychylenia w lewo i prawo}
\end{figure}

\forceindent Korzystając z wykresu po lewej stronie, odczytamy przybliżone wartości $F_{ED}$ dla natężenia prądu o wartości 0,2 A. Odczytane wartości znajdują się w tabeli poniżej: 

\begin{table}[!h]
\centering
\begin{tabular}{|c|c|}
\hline
Liczba zwojów & Wartość siły $F_{ED}$ [N]\\
\hline
5 & 0,036\\
 \hline 
10 & 0,074\\ 
 \hline 
15 & 0,107\\ 
 \hline 
20 & 0,142\\ 
 \hline 
25 & 0,171\\ 
 \hline 
\end{tabular}
\caption{Wartość siły $F_{ED}$ w zależności od wychylenia}
\end{table}

\forceindent Wartości z tabeli przedstawione na wykresie :

\begin{figure}[!h]
\centering
\begin{tikzpicture}
\begin{axis}[
xlabel={Natężenie prądu $I$ [A]},
ylabel={Wartość siły $F_{ED}$ [N]},
xmin=0,xmax=30,
ymin=0,ymax=0.2,
y tick label style={/pgf/number format/fixed},
legend pos=south east,
ymajorgrids=true,
grid style=dashed]

\addplot[smooth, tension={0.3}, red, mark=*, mark size = {0.5pt}]
coordinates {
(5, 0.036)
(10, 0.074) 
(15, 0.107) 
(20, 0.142) 
(25, 0.171) 
};

\legend{5 zwojów}
\end{axis}
\end{tikzpicture}
\caption{Zależność wartości siły $F_{ed}$ od liczby uzwojeń dla natężenia 0,2A i wychylenia w lewo}
\end{figure}



\forceindent Wyliczymy teraz, korzystając z metody regresji liniowej, współczynnik nachylenia prostej odpowiadającej zależności siły elektrodynamicznej od natężenia prądu przepływającego przez uzwojenia ramki (dla 10 zwojów) wychylonej w prawo.\\
Przyjmujemy, że $F_{ED} = y$, $I = x$ i wzorami:
$$
a=\frac{n\Sigma x_i y_i - \Sigma x_i \Sigma y_i}{n\Sigma x_i^2 - (\Sigma x_i)^2},
$$
$$
b=\frac{\Sigma x_i^2 \Sigma y_i - \Sigma x_i \Sigma x_i y_i}{n \Sigma x_i^2 - (\Sigma x_i)^2},
$$
wyznaczamy współczynnik $a$ oraz punkt $b$ przecięcia prostej z osią OY:
$$
a = 0,356601115 
$$
$$
b = 0,0048569 
$$
Błąd wyznaczenia wielkości $a$:\\
$$ \Delta a = \sqrt{\frac{n(\Sigma y_i ^2 - a \Sigma x_i y_i - b\Sigma y_i)}{(n-2)(n \Sigma x_i ^2 - (\Sigma x_i)^2)}} = 0,003340024$$
Następnie korzystając z równania:
\begin{equation}
B = \frac{a_R}{n\cdot a}
\end{equation}
gdzie $a_R$ jest obliczonym przez nas wyżej współczynnikiem $a$, $n$ jest ilością zwojów zaś $a$ jest długością dolnego boku ramki i wynosi $a = (13,0 \pm 0,2) cm$,
obliczamy wartość indukcji pola magnetycznego $B$, która wynosi:
$$ B = \frac{0,356601115}{10\cdot 0,13} = 0,27430855 T $$

Błąd wyznaczenia $B$ wyznaczamy za pomocą różniczki logarytmicznej:
$$ \Delta B = \left ( \frac{\Delta a_R}{a_R} + \frac{\Delta a}{a}  \right ) B$$
co po podstawieniu odpowiednich wartości $\Delta a_R$, $a_R$, $\Delta a$, $a$ i $B$ daje nam:
$$ \Delta B = 0,006789381 T $$


\forceindent Zatem ostateczne wartości $a_R$ i $B$ wyglądają następująco:

\begin{table}[!h]
\centering
\begin{tabular}{|cc||c|c|}
\multicolumn{1}{c}{} & \multicolumn{1}{c}{$a_R$} & \multicolumn{1}{c}{$B$}\\
\hline
wynik obliczeń & $0,356601115$ & $0,27430855$ \\
\hline
dokładność & $0,003340024$ & $0,006789381$\\
\hline
po zaokrągleniu & $0,357 \pm 0,003$  & $(0,274 \pm 0,007) T $\\
\hline
\end{tabular}
\caption{Współczynnik nachylenia prostej $a_R$ i wartość indukcji pola magnetycznego $B$ wraz z dokładnościami $\Delta a_R$ i $\Delta E_A$}
\end{table}

\section*{Wnioski}
\forceindent Wyniki przeprowadzonego doświadczenia istotnie potwierdzają tezę, że siła elektrodynamiczna działająca na ramkę w polu magnetycznym, przez którą płynie prąd, jest wprost proporcjonalna do natężenia prądu płynącego przez tą ramkę i ilości zwojów, z której się składa.  \\
\forceindent Natomiast obliczona wartość indukcji pola magnetycznego, w którym znajdowała się ramka jest tak naprawdę wartością średnią, ponieważ pole między biegunami magnesu nie jest w pełni jednorodne. \\
\forceindent Wszelkie błędy wynikają w dużej mierze z niedokładności pomiarów.


\end{document}
