\documentclass[10pt,a4paper]{article}

\usepackage{polski}
\usepackage[utf8]{inputenc}

\usepackage[polish]{babel}
\usepackage{hhline}
\usepackage{pgfplots}

\usepackage{multicol}
%\usepackage{slashbox}
\usepackage{graphicx}
\usepackage{caption}
\usepackage{subcaption}
\usepackage{colortbl}

\usepackage{geometry}
\geometry{a4paper, total={170mm,257mm}, left=20mm, top=20mm }

\author{Sebastian Maciejewski 132275 i Jan Techner 132332}
\title{Wyznaczanie stałej siatki dyfrakcyjnej - \\ doświadczenie 303 (sala 221)}
\date{19 stycznia 2017}
\setlength{\parindent}{0pt}
\newcommand{\forceindent}{\leavevmode{\parindent=3em\indent}}
\begin{document}
\maketitle
\section{Wstęp teoretyczny}
\forceindent Światło jest falą elektromagnetyczną, czyli falą polegającą na rozchodzeniu się w przestrzeni zmian natężenia pola elektrycznego i magnetycznego. Ponieważ światło 
zachowuje się jak fala i jak cząsteczka, mówimy o jego korpuskularno-falowej naturze. Jako fala, światło podlega zasadzie Huygensa, która mówi, że każdy punkt, do którego dochodzi fala
staje się nowym źródłem fali kulistej, co jest przyczyną zjawiska dyfrakcji. Dyfrakcję światła obserwujemy wtedy, gdy przechodzi ono przez niewielki otwór w nieprzezroczystej przeszkodzie, 
w szczególności wtedy, gdy szerokość szczeliny jest w przybliżeniu równa lub mniejsza od długości fali światła padającego na szczelinę.
Taka sytuacja ma miejsce gdy zastosuje się siatki dyfrakcyjne - wówczas zjawisko dyfrakcji zależy od stałej siatki, będącej parametrem opisującym rozstaw szczelin siatki.

\subsection*{Opis doświadczenia}
\forceindent Doświadczenie polega na zapisywaniu położenia prążków: zerowego i wyższych rzędów dla poszczególnych siatek dyfrakcyjnych. Zapisywaną wartością jest w rzeczywistości,
po odjęciu wartości odczytanej dla prążka zerowego, kąt $\alpha$ ugięcia dla prążka każdego z rzędów. Na jego podstawie można obliczyć stałą siatki, która wyrażona jest wzorem:
\begin{equation}
d=\frac{m\lambda}{sin\alpha}
\end{equation}
gdzie $m$ rząd maksimum, $\lambda$ to długość fali (dla używanej w doświadczeniu lampy sodowej wartość ta wynosi $\lambda = 589,6nm$), zaś $sin\alpha$ to kąt ugięcia dla prążka rzędu $m$.\\
Dzięki obliczeniu wartości $d$ dla każdego rzędu, można później obliczyć średnią wartość stałej $d$ dla każdej z badanych siatek.//

\newpage
\section{Wyniki pomiarów i obliczenia}

Prążek zerowy został oczytany przy ustawieniu kolimatora pod kątem $\alpha = 177,7167^{\circ}$.\\

Wyniki pomiarów dla kąta $\alpha[^\circ]$ wraz z obliczoną na ich podstawie wartością stałej siatki dyfrakcyjnej $d [m]$ oraz jej niepewnością $\Delta d [m]$ dla poszczególnych siatek przedstawiono dla wygody w jednej tabeli :

\begin{center}
\begin{tabular}{|c|c|c|c|c|c|c|}
\hline
\multicolumn{7}{|c|}{Siatka A}\\
\hline
& \multicolumn{3}{c}{W lewo} & \multicolumn{3}{|c|}{W prawo}\\
\hline
Rząd & $\alpha_{lewo}$ & $d$ &$\Delta d$ &$\alpha_{prawo}$&$d$&$\Delta d$\\
\hline
0&0&0&0&0&0&0\\ 
\hline
1&2,6667&0,00001267&0,000004535&2,6167&0,00001291&0,000004710\\ 
\hline 
2&5,3667&0,00001261&0,000002237&5,3500&0,00001265&0,000002251\\ 
\hline 
3&8,0333&0,00001266&0,000001495&8,0167&0,00001268&0,000001501\\ 
\hline 
4&10,7833&0,00001261&0,000001103&10,7000&0,00001270&0,000001120\\ 
\hline 
5&13,4500&0,00001267&0,0000008833&13,5000&0,00001263&0,0000008767\\ 
\hline 
6&16,1333&0,00001273&0,0000007335&16,3000&0,00001260&0,0000007184\\ 
\hline 
7&18,9333&0,00001272&0,0000006180&19,0833&0,00001262&0,0000006082\\ 
\hline 
8&21,7833&0,00001271&0,0000005301&21,9833&0,00001260&0,0000005202\\ 
\hline 
9&24,5500&0,00001277&0,0000004660&24,9667&0,00001257&0,0000004500\\ 
\hline 
10&27,5167&0,00001276&0,0000004083&27,9833&0,00001257&0,0000003942\\ 
\hline 
11&30,5333&0,00001277&0,0000003607&31,1167&0,00001255&0,0000003465\\ 
\hline 
12&33,6500&0,00001277&0,0000003197&34,4000&0,00001252&0,0000003048\\ 
\hline 
13&&&&37,7167&0,00001253&0,0000002700\\ 
\hline 
14&&&&41,3333&0,00001250&0,0000002368\\ 
\hline 
\end{tabular}
\end{center}

\begin{center}
\begin{tabular}{|c|c|c|c|c|c|c|}
\hline
\multicolumn{7}{|c|}{Siatka B}\\
\hline
& \multicolumn{3}{c}{W lewo} & \multicolumn{3}{|c|}{W prawo}\\
\hline
Rząd & $\alpha_{lewo}$ & $d$ &$\Delta d$ &$\alpha_{prawo}$&$d$&$\Delta d$\\
\hline
0&0&0&0&0&0&0\\ 
\hline
1&6,7667 &0,000005004&0,0000007029&6,7333&0,000005029&0,0000007098\\ 
\hline 
2&13,5833&0,000005021&0,0000003463&13,6500&0,000004997&0,0000003429\\ 
\hline 
3&20,7000&0,000005004&0,0000002207&20,5833&0,000005031&0,0000002233\\ 
\hline 
 
\end{tabular}
\end{center}

\begin{center}
\begin{tabular}{|c|c|c|c|c|c|c|}
\hline
\multicolumn{7}{|c|}{Siatka C}\\
\hline
& \multicolumn{3}{c}{W lewo} & \multicolumn{3}{|c|}{W prawo}\\
\hline
Rząd & $\alpha_{lewo}$ & $d$ &$\Delta d$ &$\alpha_{prawo}$&$d$&$\Delta d$\\
\hline
0&0&0&0&0&0&0\\ 
\hline
1 &13,5833 &0,000002510 & 0,0000001732 & 13,6833 & 0,000002492 & 0,0000001706\\ 
\hline 
2 &27,9167 &0,000002519 & 0,00000007923 & 28,4333 & 0,000002477 & 0,00000007623\\ 
\hline 
3 &44,3167 &0,000002532 & 0,00000004322 & & &\\ 
\hline 

 
\end{tabular}
\end{center}

\begin{center}
\begin{tabular}{|c|c|c|c|c|c|c|}
\hline
\multicolumn{7}{|c|}{Siatka D}\\
\hline
& \multicolumn{3}{c}{W lewo} & \multicolumn{3}{|c|}{W prawo}\\
\hline
Rząd & $\alpha_{lewo}$ & $d$ &$\Delta d$ &$\alpha_{prawo}$&$d$&$\Delta d$\\
\hline
0&0&0&0&0&0&0\\ 
\hline
1&20,6833&0,000001669& 0,00000007369 & 20,6500& 0,000001672 &0,00000007394\\ 
\hline 
2&44,9000&0,000001671& 0,00000002794 & 45,0167& 0,000001667 &0,00000002777\\ 
\hline 
 
\end{tabular}
\end{center}

\begin{center}
\begin{tabular}{|c|c|c|c|c|c|c|}
\hline
\multicolumn{7}{|c|}{Siatka E}\\
\hline
& \multicolumn{3}{c}{W lewo} & \multicolumn{3}{|c|}{W prawo}\\
\hline
Rząd & $\alpha_{lewo}$ & $d$ &$\Delta d$ &$\alpha_{prawo}$&$d$&$\Delta d$\\
\hline
0&0&0&0&0&0&0\\ 
\hline
1&23,2333&0,000001495& 0,00000005803 &23,2333& 0,000001495&0,00000005803\\ 
\hline 
2&51,7833&0,000001501& 0,00000001970 &52,3500& 0,000001489&0,00000001915 \\ 
\hline 

\end{tabular}
\end{center}

\newpage

\forceindent W tabelach powyżej dla każdego pomiaru została przedstawiona niepewność $\Delta d$, związana z niepewnością odczytu kąta $\alpha$. Dla stosowanych do pomiaru urządzeń, 
niepewność wynosiła 1 minutę kątową czyli w przybliżeniu $\Delta \alpha = 0,0167^\circ$. Równania $(1)$ nie można przedstawić w postaci iloczynowej, więc do wyznaczenia niepewności pomiarowej zastosowano metodę różniczki zupełnej :

$$
\Delta d = | \frac{\partial d}{\partial \alpha} \cdot \Delta \alpha | 
$$ 
$$
\frac{\partial d}{\partial \alpha} = \frac{-m\lambda \cdot cos\alpha}{sin^2 \alpha}
$$.


\forceindent Znając wartości stałych $d$ dla każdego rzędu w każdej siatce, można obliczyć stałą $d$ każdej z siatek jako średnią $d$ dla poszczególnych rzędów. Natomiast niepewność pomiarowa dla każdej siatki została wyznaczona jako największa niepewność ze wszystkich pomiarów dla danej siatki. Wyznaczone wartości to:
\begin{center}
\begin{tabular}{|c|c|c|}
\hline
Siatka & Stała $d$ & Niepewność $\Delta d$\\
\hline
A & $1,266 \cdot 10^{-5} m$ & $4,710 \cdot 10^{-6} m$\\
\hline
B & $5,014 \cdot 10^{-6} m$ & $7,098 \cdot 10^{-7} m$ \\
\hline
C & $2,506 \cdot 10^{-6} m$ & $1,732 \cdot 10^{-7} m$ \\
\hline
D & $1,670 \cdot 10^{-6} m$ & $7,394 \cdot 10^{-8} m$\\
\hline
E & $1,495 \cdot 10^{-6} m$ & $5,803 \cdot 10^{-8} m$\\
\hline
\end{tabular}
\end{center}



Ostateczne wartości $d$ po zaokrągleniu wyglądają następująco:

\begin{center}
\begin{tabular}{|c|c|c|}
\hline
Siatka & Stała $d$ & Niepewność $\Delta d$\\
\hline
A & $1,3 \cdot 10^{-5} m$ & $5,0 \cdot 10^{-6} m$\\
\hline
B & $5,0 \cdot 10^{-6} m$ & $7,0 \cdot 10^{-7} m$ \\
\hline
C & $2,5 \cdot 10^{-6} m$ & $1,7 \cdot 10^{-7} m$\\
\hline
D & $1,7 \cdot 10^{-6} m$ & $8,0 \cdot 10^{-8} m$\\
\hline
E & $1,5 \cdot 10^{-6} m$ & $6,0 \cdot 10^{-8} m$\\
\hline
\end{tabular}
\end{center}

\section*{Wnioski}

Na podstawie wyznaczonych wartości można zauważyć, że im mniejsza jest stała siatki dyfrakcyjnej, czyli im mniejsza jest odległość między środkami poszczególnych szczelin, tym dalej od siebie są prążki interferencyjne, co zgadza się z \\ \\??? WTF, kompletnie nie mam pomysłu na te wnioski...... 	 


\end{document}
