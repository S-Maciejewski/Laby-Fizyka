\documentclass[10pt,a4paper]{article}

\usepackage{polski}
\usepackage[utf8]{inputenc}

\usepackage[polish]{babel}
\usepackage{hhline}
\usepackage{pgfplots}

\usepackage{geometry}
\geometry{a4paper, total={170mm,257mm}, left=20mm, top=20mm }

\author{Sebastian Maciejewski (132275) i Jan Techner 132332}
\title{Wyznaczanie współczynnika rozszerzalności liniowej ciał stałych - \\ doświadczenie 103 (sala 217)}
\date{10 listopada 2017}
\setlength{\parindent}{0pt}
\newcommand{\forceindent}{\leavevmode{\parindent=3em\indent}}
\begin{document}
\maketitle
\section{Wstęp teorytyczny}
\forceindent Zmiana temperatury ciała z reguły powoduje zmianę jego wymiarów liniowych. Elementarny przyrost temperatury $dT$ ciała o długości $l$ powoduje przyrost długości $dl$, który jest określony wzorem: 
\begin{equation}
dl = \alpha \, l \, dT.
\end{equation}
\forceindent Współczynnik $\alpha$ nazywany jest współczynnikiem rozszerzalności liniowej. Jego wartość liczbowa jest równa względnemu przyrostowi długości $dl/l$ spowodowanemu zmianą temperatury o $1^{\circ}$C i zależy od rodzaju ciała, a także od temperatury. Zależność współczynnika $\alpha$ od temperatury powoduje, że długość ciała jest na ogół nieliniową funkcją temperatury. Jednakże w zakresie niewielkich zmian temperatury można przyjąć, że współczynnik $\alpha$ jest stały, a długość jest wprost proporcjonalna do temperatury. W takim przypadku wzór $(1)$ można zastąpić wzorem: 
\begin{equation}
l - l_{0} = \alpha_{sr} \, l_{0} \, \Delta T,
\end{equation}
który znacznie ułatwia obliczenie długości w dowolnej temperaturze. \\ \newline
\forceindent Przyczyny zjawiska rozszerzalności cieplnej są związane ze stukturą mikroskopową ciał. Ciała stałe są zbudowane z atomów (jonów) rozłożonych regularnie w przestrzeni i tworzących sieć krystaliczą. Atomy są ze sobą powiązane siłami pochodzenia elektrycznego, co uniemożliwia im trwałą zmianę położenia. Dostarczona do kryształu energia cieplna wywołuje drgania atomów wokół położeń równowagi, a amplituda tych drgań rośnie wraz ze wzrostem temperatury. Częstotliwość drgań atomów sięga rzędu $10^{13}$ Hz. W tej sytuacji ciężko jest określić odległośc między atomami, a pojęcie to ma sens tylko jako odległość pomiędzy środkami drgań sąsiednich atomów. \\ \newline
\forceindent Gdyby energia kinetyczna atomów była równa zeru, znajdowałyby się one w odległości $r_0$ od siebie, a przy tej odległości energia potencjalna ma swoje minimum. W rzeczywistości jednak atomy wykonują drgania wokół położeń równowagi, tnz. mają pewną energię kinetyczną, zależną od temperatury. Wskutek asymetrii krzywej potencjalnej średnia odległość między cząsteczkami nie będzie się pokrywać z wartością $r_0$, ale będzie rosła wraz ze wzrostem temperatury. \\ \newline
\forceindent Z powyższego opisu wynika, że podczas wzrostu temperatury rośnie nie tylko amplituda drgań atomów, ale także średnia odległość między nimi, co rzutuje na makroskopowe wydłużenie ciała zwane rozszerzalnością cieplną.  

\subsection*{Opis doświadczenia}

\forceindent Badane ciała, w naszym przypadku 3 pręty (stalowy, mosiężny i miedziany), umieszczamy w płaszczu wodnym połączonym z termostatem. Jeden koniec każdego z prętów umieszczony jest w uchwycie, natomiast drugi przesuwa się w miarę podgrzewania. Wydłużenie każdego pręta mierzymy czujnikiem mikrometrycznym, a jego temperaturę termometrem elektrycznym.  

\section{Wyniki pomiarów}
Początkowa długość i temperatura prętów została zamierzona na początku i wynosiła:\\
23,4 $^{\circ}$C i 72,3 cm dla Stali,\\
23,8 $^{\circ}$C i 71,2 cm dla Mosiądzu,\\
23,8 $^{\circ}$C i 72,35 cm dla Miedzi.\\

\newpage
Poniższa tabela ukazuje zmiany długości prętów pod wpływem temperatury.\\
Dokładność pomiarów temperatury to $\Delta t = \pm 0,1 ^{\circ}C$, zaś pomiarów długości to $\Delta l = \pm 0,01 mm$.

\begin{center}
\begin{tabular}{|c|c||c|c||c|c|}
\multicolumn{2}{c}{Stal} & \multicolumn{2}{c}{Mosiądz} & \multicolumn{2}{c}{Miedź}\\
\hhline{|=|==|==|=|}
\textbf{t ($^{\circ}$C)} & \textbf{$\Delta$ l (mm)} & \textbf{t ($^{\circ}$C)} & \textbf{$\Delta$ l (mm)} & \textbf{t ($^{\circ}$C)} & \textbf{$\Delta$ l (mm)}\\
\hline
27,5 & 0,04 & 27,7 & 0,07 & 27,4 & 0,06\\
\hline
33,2 & 0,11 & 33,3 & 0,19 & 32,6 & 0,15\\
\hline
38,6 & 0,16 & 38,3 & 0,26 & 38,2 & 0,23\\
\hline
43,1 & 0,21 & 43,0 & 0,34 & 43,0 & 0,30\\
\hline
48,0 & 0,25 & 48,3 & 0,44 & 48,3 & 0,38\\
\hline
53,2 & 0,32 & 53,0 & 0,53 & 53,0 & 0,47\\
\hline
58,2 & 0,37 & 58,1 & 0,63 & 58,0 & 0,54\\
\hline
53,3 & 0,28 & 53,1 & 0,50 & 53,2 & 0,42\\
\hline
48,3 & 0,23 & 48,4 & 0,42 & 48,4 & 0,34\\
\hline
43,3 & 0,23 & 43,2 & 0,33 & 43,1 & 0,26\\
\hline
38,5 & 0,13 & 38,4 & 0,26 & 38,4 & 0,19\\
\hline
33,2 & 0,08 & 33,3 & 0,17 & 32,2 & 0,12\\
\hline
28,1 & 0,02 & 28,1 & 0,08 & 28,2 & 0,04\\
\hline

\end{tabular}
\end{center}

\section{Opracowanie wyników}


\end{document}