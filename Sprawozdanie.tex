\documentclass[10pt,a4paper]{article}

\usepackage{polski}
\usepackage[utf8]{inputenc}

\usepackage[polish]{babel}
\usepackage{hhline}
\usepackage{pgfplots}

\usepackage{geometry}
\geometry{a4paper, total={170mm,257mm}, left=20mm, top=20mm }

\author{Sebastian Maciejewski 132275 i Jan Techner 132332}
\title{Badanie momentu bezwładności - \\ doświadczenie 104 (sala 217)}
\date{24 listopada 2017}
\setlength{\parindent}{0pt}
\newcommand{\forceindent}{\leavevmode{\parindent=3em\indent}}
\begin{document}
\maketitle
\section{Wstęp teorytyczny}
\forceindent W opisie dynamiki ruchu postępowego pojawia się pojęcie bezwładności związane z masą \textit{m}
poruszającego się ciała. W przypadku ruchu obrotowego znajomość masy ciała jest niewystarczająca, istotny
jest również jej przestrzenny rozkład względem osi obrotu. Wielkością fizyczną zawierającą informacje o
masie ciała i jej przestrzennym rozkładzie względem osi obrotu jest moment bezwładności \textit{I}.\\
\forceindent Dla pojedynczego punktu materialnego o masie m wirującego wokół osi oddalonej od niego o
odległość \textit{r} możemy zapisać następującą zależność na moment bezwładności:
\begin{equation}
I=mr^2
\end{equation}

\subsection*{Opis doświadczenia}

\forceindent W ćwiczeniu zostaną wyznaczone momenty bezwładności stalowego pręta oraz dysku. Dodatkowym
zadaniem będzie eksperymentalne potwierdzenie twierdzenia Steinera. Do badań posłuży wahadło skrętne
złożone ze stabilnej podstawy oraz pionowej osi osadzonej na łożyskach o bardzo małym tarciu. Oś oraz
podstawa połączone są przy pomocy spiralnej sprężyny, która umożliwia wahania skrętne. Na końcu osi
znajduje się śruba umożliwiająca mocowanie na niej brył.\\
\forceindent Wahadło skrętne jest szczególnym przypadkiem wahadła fizycznego. Jeżeli założymy, że wychylenia
wahadła są niewielkie (do około $180^{\circ}$) oraz zaniedbamy siły oporu, jego ruch można opisać jako ruch
harmoniczny prosty. W takim przypadku okres T drgań wahadła można zapisać następująco:
\begin{equation}
T = 2 \pi \sqrt{\frac{I}{D}}
\end{equation}
Gdzie $D$ jest parametrem charakterystycznym dla danej sprężyny - jej momentem kierującym.

\newpage
\section{Wyniki pomiarów}
TODO tabelka z wynikami
\section{Opracowanie wyników}
TODO wykresy

\forceindent W celu obliczenia momentów bezwładności z danych pomiarowych posłużymy się równaniem :
\begin{equation}
T^2 = \frac{8 \pi^2 m_c}{D}r^2 + T_p^2
\end{equation}
gdzie $T$ jest okresem drgań naszego wahadła, $m_c$ jest zmierzoną masą ciężarka, $r$ jest odległością ciężarków (lub środka dysku) od osi obrotu, zaś $D$ jest momentem kierującym.\\
\forceindent Równanie $(3)$ oznacza, że zależność kwadratu okresów $T$ od kwadratów odległości $r$ jest liniowa (co w przybliżeniu widać na wykresie). Można zatem policzyć współczynnik nachylenia prostej przy pomocy metody regresji liniowej (gdzie $x = r^2$ i $y = T^2$). Współczynnik $a$ wyraża się wzorem:

\begin{equation}
a=\frac{n\Sigma x_i y_i - \Sigma x_i \Sigma y_i}{n\Sigma x_i^2 - (\Sigma x_i)^2}.
\end{equation}

Znając $a$ możemy obliczyć moment kierujący dla tego wahadła, który wyraża się jako:
\begin{equation}
D = \frac{8\pi ^2 m_c}{a}
\end{equation}

Obliczone na podstawie pomiarów wartości przedstawione są, wraz z dokładnościami, w tabeli poniżej.

\begin{table}[!h]
\centering
\begin{tabular}{|cc||c|}
\multicolumn{1}{c}{} & \multicolumn{1}{c}{$a$} & \multicolumn{1}{c}{$D$}\\
\hline
pomiar & 24366,1434802235$\pm$& 0,000839271932206056$\pm$\\
\hline
dokładność & zmyślona & zmyślona\\
\hline
po zaokrągleniu & ileś tam  & ileś tam \\
\hline
\end{tabular}
\caption{Współczynnik nachylenia linii $a$ i moment kierujący $D$ wraz z dokładnościami $\Delta a$ i $\Delta D$}
\end{table}

TODO\\
\forceindent Na wykresach, które były przedstawione powyżej, widać, że wykres funkcji zależności $T^2 = f(r^2)$ jest w przybliżeniu prostą.
Widać także, że różnica pomiędzy teoretycznie obliczoną (przy pomocy twierdzenia Steinera) wartością momentu bezwładności a wartością zmierzoną jest w większości przypadków mniejsza niż dokładność obliczona na podstawie możliwości przyrządów pomiarowych. Część wyników nieznacznie odstaje od reszty (o więcej niż jeden raz wartość dokładności pomiaru), co prawdopodobnie jest spowodowane błędem wynikającym z opóźnionego czasu reakcji osoby dokonującej pomiarów czasu. Wartość takiego błędu ciężko oszacować, jednak widać, że dla mniejszych wartości $T$ jest on mniej zauważalny, co związane jest z tym, że we wzorze $T$ jest podnoszone do kwadratu (błąd pomiarowy staje się bardziej znaczącym czynnikiem).\\
Mimo tych drobnych błędów uważamy, że nasze pomiary i obliczenia dowodzą w sposób doświadczalny zasadności twierdzenia Steinera.



\end{document}