\documentclass[10pt,a4paper]{article}

\usepackage{polski}
\usepackage[utf8]{inputenc}

\usepackage[polish]{babel}
\usepackage{hhline}
\usepackage{pgfplots}

\usepackage{geometry}
\geometry{a4paper, total={170mm,257mm}, left=20mm, top=20mm }

\author{Sebastian Maciejewski 132275 i Jan Techner 132332}
\title{Badanie momentu bezwładności - \\ doświadczenie 104 (sala 217)}
\date{24 listopada 2017}
\setlength{\parindent}{0pt}
\newcommand{\forceindent}{\leavevmode{\parindent=3em\indent}}
\begin{document}
\maketitle
\section{Wstęp teorytyczny}


\subsection*{Opis doświadczenia}

\section{Wyniki pomiarów}

\section{Opracowanie wyników}
\forceindent Dla zależności:
\begin{equation}
ln(1/R) = f(1/T)
\end{equation}  
wyliczymy teraz, korzystając z metody regresji liniowej, współczynnik nachylenia prostej.\\
Przyjmujemy, że $ln(1/R) = y$ i $1/T = x$.
Posługując się metodą regresji liniowej opisaną wzorem:
\begin{equation}
a=\frac{n\Sigma x_i y_i - \Sigma x_i \Sigma y_i}{n\Sigma x_i^2 - (\Sigma x_i)^2},
\end{equation}
wyznaczamy współczynnik nachylenia prostej $a$, oraz jego niepewność. 
\begin{equation}
a = -3869,397 \left[\frac{K}{\Omega}\right]
\end{equation}
Następnie korzystając z równania:
\begin{equation}
a = \frac{E_A}{2k} \Rightarrow E_A = 2ak
\end{equation}
obliczamy energię aktywacji ($E_A$), która wynosi:
$$ E_A = -1,068 * 10^-19 \frac{J}{K} = -0,667 \frac{eV}{K} $$

Błąd wyznaczenia wielkości $a$:\\
$$ \Delta a = \sqrt{\frac{n(\Sigma y_i ^2 - a \Sigma x_i y_i - b\Sigma y_i)}{(n-2)(n \Sigma x_i ^2 - (\Sigma x_i)^2)}} = $$


\forceindent Zatem ostateczne wartości $a$ i $E_A$ wyglądają następująco:

\begin{table}[!h]
\centering
\begin{tabular}{|cc||c|c|}
\multicolumn{1}{c}{} & \multicolumn{1}{c}{$a$} & \multicolumn{1}{c}{$E_A [\frac{J}{K}]$} & \multicolumn{1}{c}{$E_A [\frac{eV}{K}]$}\\
\hline
pomiar & $-3869,39702854943$ & $-1,068 * 10^-19$ & $-0,667$\\
\hline
dokładność & TODO & TODO & TODO\\
\hline
po zaokrągleniu & TODO  & TODO & TODO\\
\hline
\end{tabular}
\caption{Współczynnik nachylenia linii $a$ i energia aktywacji $E_A$ wraz z dokładnościami $\Delta a$ i $\Delta E_A$}
\end{table}




\end{document}