\documentclass[10pt,a4paper]{article}

\usepackage{polski}
\usepackage[utf8]{inputenc}

\usepackage[polish]{babel}
\usepackage{hhline}
\usepackage{pgfplots}

\usepackage{slashbox}
\usepackage{graphicx}
\usepackage{caption}
\usepackage{subcaption}

\usepackage{geometry}
\geometry{a4paper, total={170mm,257mm}, left=20mm, top=20mm }

\author{Sebastian Maciejewski 132275 i Jan Techner 132332}
\title{Badanie oddziaływania pola magnetycznego na przewodnik z prądem - \\ doświadczenie 204 (sala 217A)}
\date{22 grudnia 2017}
\setlength{\parindent}{0pt}
\newcommand{\forceindent}{\leavevmode{\parindent=3em\indent}}
\begin{document}
\maketitle
\section{Wstęp teoretyczny}
\forceindent Oddziaływanie pola magnetycznego na przewodnik z prądem jest zjawiskiem powszechnie wykorzystywanym w technice. W licznych doświadczeniach z cząsteczkami obdarzonymi ładunkiem elektrycznym poruszającymi się w polu magnetycznym zaobserwowano występowanie siły powodującej zakrzywienie ich toru. Siłę tę jako pierwszy opisał Holenderski fizyk Hendrik Lorentz, a konsekwencją występowania siły Lorentza jest siła działająca w polu magnetycznym na przewodnik z prądem. Siła ta bierze się z tego, że w odcinku o długości $l$ w danej chwili przepływa $n$ elektronów o ładunku $e$ i średniej prędkości unoszenia $v_u$ i na każdy elektron działa siła Lorentza.

\subsection*{Opis doświadczenia}
\forceindent Doświadczenie polega na zbadaniu siły elektrodynamicznej działającej na ramkę w polu magnetycznym. Chcąc wyznaczyć siłę elektrodynamiczną działającą na dolny bok ramki, musimy rozważyć momenty sił występujące po wychyleniu ramki z położenia równowagi. Momenty te będą związane z siłami: grawitacji oraz elektrodynamiczną. \\
Szukana przez nas siła będzie miała postać $$F_ED = \frac{1}{2} tg(\alpha)mg,$$ gdzie $m$ to masa ramki i $\alpha$ to kąt odchylenia.\\
Ponieważ widać, że $tg(\alpha)$ będzie równy stosunkowi długości odcinka $x$ odczytanej ze skali do odległości osi obrotu od środka skali $d$, można powyższy wzór przedstawić jako
$$ F_{ED} = \frac{mgx}{2d}, $$ a z racji tego, że wielkości $m$, $g$ i $d$ są stałe finalna postać wzoru to (po podstawieniu $c=\frac{mg}{2d}$):
$$ F_{ED} = cx $$


\newpage
\section{Wyniki pomiarów}

Wyniki pomiarów natężenia prądu $I[A]$ w zależności od wychylenia i liczby zwojów\\

\begin{center}
\begin{tabular}{|c|c|c|c|c|c|c|c|c|c|c|}
%\multicolumn{1}{c}{zwoje} & \multicolumn{10}{c}{wychylenie w $m$}
\hline
\backslashbox{Zwoje}{Wychylenie ($m$)}&0,01&0,02&0,03&0,04&0,05&0,06&0,07&0,08&0,09&0,1\\ 
 \hline 
5&0,128&0,281&0,421&0,566&0,706&0,859&1,008&1,118&1,281&1,426\\ 
 \hline 
-5&0,145&0,289&0,426&0,596&0,738&0,882&1,057&1,213&1,381&1,557\\ 
 \hline 
10&0,065&0,141&0,208&0,277&0,351&0,432&0,501&0,592&0,652&0,732\\ 
 \hline 
-10&0,075&0,141&0,218&0,301&0,374&0,454&0,529&0,616&0,694&0,781\\ 
 \hline 
15&0,046&0,091&0,143&0,195&0,242&0,282&0,337&0,39&0,444&0,495\\ 
 \hline 
-15&0,047&0,098&0,139&0,198&0,245&0,307&0,354&0,417&0,478&0,525\\ 
 \hline 
20&0,037&0,075&0,102&0,143&0,186&0,213&0,261&0,298&0,341&0,381\\ 
 \hline 
-20&0,036&0,071&0,112&0,152&0,187&0,227&0,278&0,315&0,369&0,419\\ 
 \hline 
25&0,02&0,049&0,088&0,118&0,144&0,175&0,202&0,241&0,275&0,308\\ 
 \hline 
-25&0,027&0,055&0,085&0,122&0,147&0,186&0,218&0,248&0,293&0,323\\ 
 \hline 
\end{tabular}
\end{center}

\section{Opracowanie wyników}

\forceindent Poniższe wykresy TODO



\forceindent Wyliczymy teraz, korzystając z metody regresji liniowej, współczynnik nachylenia prostej odpowiadającej zależności siły elektrodynamicznej od natężenia prądu przepływającego przez uzwojenia ramki (dla 10 zwojów) wychylonej w prawo.\\
Przyjmujemy, że $F_{ED} = y$, $I = x$ i wzorami:
\begin{equation}
a=\frac{n\Sigma x_i y_i - \Sigma x_i \Sigma y_i}{n\Sigma x_i^2 - (\Sigma x_i)^2},
\end{equation}
\begin{equation}
b=\frac{\Sigma x_i^2 \Sigma y_i - \Sigma x_i \Sigma x_i y_i}{n \Sigma x_i^2 - (\Sigma x_i)^2},
\end{equation}
wyznaczamy współczynnik $a$ oraz punkt $b$ przecięcia prostej z osią OY:
\begin{equation}
a = 0,356601115 
\end{equation}
\begin{equation}
b = 0,0048569 
\end{equation}
Błąd wyznaczenia wielkości $a$:\\
$$ \Delta a = \sqrt{\frac{n(\Sigma y_i ^2 - a \Sigma x_i y_i - b\Sigma y_i)}{(n-2)(n \Sigma x_i ^2 - (\Sigma x_i)^2)}} = 0,003340024$$
Następnie korzystając z równania:
\begin{equation}
B = \frac{a_R}{na}
\end{equation}
gdzie $a_R$ jest obliczonym przez nas wyżej współczynnikiem $a$, $n$ jest ilością zwojów zaś $a$ jest długością dolnego boku ramki i wynosi $a = (13,0 \pm 0,2) cm$,
obliczamy wartość indukcji pola magnetycznego $B$, która wynosi:
$$ B = \frac{0,356601115}{10*0,13} = 0,27430855 T $$

Błąd wyznaczenia $B$ wyznaczamy za pomocą różniczki logarytmicznej:
$$ \Delta B = \left ( \frac{\Delta a_R}{a_R} + \frac{\Delta a}{a}  \right ) B$$
co po podstawieniu odpowiednich wartości $\Delta a_R$, $a_R$, $\Delta a$, $a$ i $B$ daje nam:
$$ \Delta B = 0,006789381 T $$


\forceindent Zatem ostateczne wartości $a_R$ i $B$ wyglądają następująco:

\begin{table}[!h]
\centering
\begin{tabular}{|cc||c|c|}
\multicolumn{1}{c}{} & \multicolumn{1}{c}{$a$} & \multicolumn{1}{c}{$B$}\\
\hline
wynik obliczeń & $0,356601115$ & $0,27430855$ \\
\hline
dokładność & $0,003340024$ & $0,006789381$\\
\hline
po zaokrągleniu & $0,357 \pm 0,003$  & $(0,274 \pm 0,007) T $\\
\hline
\end{tabular}
\caption{Współczynnik nachylenia prostej $a$ i wartość indukcji pola magnetycznego $B$ wraz z dokładnościami $\Delta a$ i $\Delta E_A$}
\end{table}

\section*{Wnioski}
\forceindent TODO


\end{document}
