\documentclass[10pt,a4paper]{article}

\usepackage{polski}
\usepackage[utf8]{inputenc}

\usepackage[polish]{babel}
\usepackage{hhline}
\usepackage{pgfplots}

\usepackage{geometry}
\geometry{a4paper, total={170mm,257mm}, left=20mm, top=20mm }

\author{Sebastian Maciejewski 132275 i Jan Techner 132332}
\title{Badanie momentu bezwładności - \\ doświadczenie 104 (sala 217)}
\date{24 listopada 2017}
\setlength{\parindent}{0pt}
\newcommand{\forceindent}{\leavevmode{\parindent=3em\indent}}
\begin{document}
\maketitle
\section{Wstęp teorytyczny}
\forceindent W opisie dynamiki ruchu postępowego pojawia się pojęcie bezwładności związane z masą \textit{m}
poruszającego się ciała. W przypadku ruchu obrotowego znajomość masy ciała jest niewystarczająca, istotny
jest również jej przestrzenny rozkład względem osi obrotu. Wielkością fizyczną zawierającą informacje o
masie ciała i jej przestrzennym rozkładzie względem osi obrotu jest moment bezwładności \textit{I}.\\
\forceindent Dla pojedynczego punktu materialnego o masie m wirującego wokół osi oddalonej od niego o
odległość \textit{r} możemy zapisać następującą zależność na moment bezwładności:
\begin{equation}
I=mr^2
\end{equation}

\subsection*{Opis doświadczenia}

\forceindent W ćwiczeniu zostaną wyznaczone momenty bezwładności stalowego pręta oraz dysku. Dodatkowym
zadaniem będzie eksperymentalne potwierdzenie twierdzenia Steinera. Do badań posłuży wahadło skrętne
złożone ze stabilnej podstawy oraz pionowej osi osadzonej na łożyskach o bardzo małym tarciu. Oś oraz
podstawa połączone są przy pomocy spiralnej sprężyny, która umożliwia wahania skrętne. Na końcu osi
znajduje się śruba umożliwiająca mocowanie na niej brył.\\
\forceindent Wahadło skrętne jest szczególnym przypadkiem wahadła fizycznego. Jeżeli założymy, że wychylenia
wahadła są niewielkie (do około $180^{\circ}$) oraz zaniedbamy siły oporu, jego ruch można opisać jako ruch
harmoniczny prosty. W takim przypadku okres T drgań wahadła można zapisać następująco:
\begin{equation}
T = 2 \pi \sqrt{\frac{I}{D}}
\end{equation}
Gdzie $D$ jest parametrem charakterystycznym dla danej sprężyny - jej momentem kierującym.

\newpage
\section{Wyniki pomiarów}
TODO tabelka z wynikami
\section{Opracowanie wyników}
TODO wykresy


\forceindent W celu obliczenia współczynnika rozszerzalności z danych pomiarowych posłużymy się równaniem :
\begin{equation}
\Delta l = \alpha_{sr}l_{0}t - \alpha_{sr}l_{0}t_{0},
\end{equation}
gdzie $t_{0}$ jest temperaturą początkową, w której długość pręta wynosi $l_{0}$.\\
\forceindent Równanie $(3)$ oznacza, że wydłużenie jest liniową funkcją temperatury (co dosyć dobrze widać na wykresach, zarówno w procesie ogrzewania jak i schładzania prętów) i że współczynnik nachylenia linii
\begin{equation}
a = \alpha_{sr}l_{0}
\end{equation}. \\
Wartość $a$ obliczamy, stosując regresję liniową do par danych ($\Delta l, T$) wyrażoną wzorem 
\begin{equation}
a=\frac{n\Sigma x_i y_i - \Sigma x_i \Sigma y_i}{n\Sigma x_i^2 - (\Sigma x_i)^2}.
\end{equation}
Jeżeli ponadto dokonamy pomiaru $l_{0}$ to równanie $(4)$ może służyć do ostatecznego obliczenia współczynnika rozszerzalności. \\
\begin{table}[!h]
\centering
\begin{tabular}{|cc||c|}
\multicolumn{1}{c}{} & \multicolumn{1}{c}{$a$} & \multicolumn{1}{c}{$D$}\\
\hline
pomiar & 24366,1434802235$\pm$& 0,000839271932206056$\pm$\\
\hline
dokładność & zmyślona & zmyślona\\
\hline
po zaokrągleniu & ileś tam  & ileś tam \\
\hline
\end{tabular}
\caption{Współczynnik nachylenia linii $a$ i moment kierujący $D$ wraz z dokładnościami $\Delta a$ i $\Delta D$}
\end{table}

\vspace{10pt}
\forceindent Następnie podstawiając otrzymane wartości $a$ i $l_0$ do wzoru $(4)$ otrzymamy następujące współczynniki rozszerzalności cieplnej : \\

\begin{table}[!h]
\centering
\begin{tabular}{|c||c||c|}
\multicolumn{1}{c}{Stal} & \multicolumn{1}{c}{Mosiądz} & \multicolumn{1}{c}{Miedź}\\
\hline
$(14,64 \pm 0,79)*10^{-6} K^{-1}$ & $(25,58 \pm 0,88)*10^{-6} K^{-1}$ & $\alpha_{sr}=(22,01 \pm 0,85)*10^{-6} K^{-1}$\\
 
\hline
\end{tabular}
\caption{Współczynnik rozszerzalności cieplnej $\alpha_{sr}$ i $\Delta \alpha_{sr}$}
\end{table}

Błąd $\Delta\alpha_{sr}$ został policzony ze wzoru $\Delta\alpha = \alpha(\frac{\Delta l}{l} + \frac{\Delta dl}{dl} + \frac{\Delta T}{T})$ dla każdego z badanych metali i zaokrąglony.\\




\end{document}