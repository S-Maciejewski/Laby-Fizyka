\documentclass[10pt,a4paper]{article}

\usepackage{polski}
\usepackage[utf8]{inputenc}

\usepackage[polish]{babel}
\usepackage{hhline}
\usepackage{pgfplots}

\usepackage{geometry}
\geometry{a4paper, total={170mm,257mm}, left=20mm, top=20mm }

\author{Sebastian Maciejewski 132275 i Jan Techner 132332}
\title{Badanie momentu bezwładności - \\ doświadczenie 104 (sala 217)}
\date{24 listopada 2017}
\setlength{\parindent}{0pt}
\newcommand{\forceindent}{\leavevmode{\parindent=3em\indent}}
\begin{document}
\maketitle
\section{Wstęp teorytyczny}
\forceindent W opisie dynamiki ruchu postępowego pojawia się pojęcie bezwładności związane z masą \textit{m} poruszającego się ciała. W przypadku ruchu obrotowego znajomość masy ciała jest niewystarczająca, istotny
jest również jej przestrzenny rozkład względem osi obrotu. Wielkością fizyczną zawierającą informacje o
masie ciała i jej przestrzennym rozkładzie względem osi obrotu jest moment bezwładności \textit{I}.\\
\forceindent Dla pojedynczego punktu materialnego o masie m wirującego wokół osi oddalonej od niego o
odległość \textit{r} możemy zapisać następującą zależność na moment bezwładności:
\begin{equation}
I=mr^2
\end{equation}
W przypadku układu N punktów materialnych sztywno połączonych ze sobą względem osi obrotu, zwanej osią bezwładności, moment bezwładności układu jest równy sumie momentów bezwładności poszczególnych punktów materialnych.
\begin{equation}
I=m_1r_{1}^{2} + m_2r_{2}^{2} + ... + m_Nr_{N}^{2} = \sum\limits_{i=1}^{N} m_{i}r_{i}^{2}.
\end{equation}
gdzie $m_{i}$ jest jest masą $i$-tego punktu materialnego, a $r_{i}$ jego odległością od osi bezwładności.
\subsection*{Twierdzenie Steinera} 
\forceindent Jeżeli chcemy obliczyć moment bezwładności względem dowolnej osi nieprzechodzącej przez środek masy bryły, przydatna staje się twierdznie Steinera. Mówi ono, że jeżeli moment bezwładności bryły sztywnej względem osi przechodzącej przez jej środek masy równa się $I_{0}$ to moment bezwładności tej bryły obracającej się względem innej osi równoległej do osi przechodzącej przez jej środek masy wynosi:
\begin{equation}
I = I_{0} + md^2,
\end{equation}
gdzie $m$ jest masą bryły, a $d$ odległością między osiami.

\subsection*{Opis doświadczenia}

\forceindent W ćwiczeniu zostaną wyznaczone momenty bezwładności stalowego pręta oraz dysku. Dodatkowym
zadaniem będzie eksperymentalne potwierdzenie twierdzenia Steinera. Do badań posłuży wahadło skrętne
złożone ze stabilnej podstawy oraz pionowej osi osadzonej na łożyskach o bardzo małym tarciu. Oś oraz
podstawa połączone są przy pomocy spiralnej sprężyny, która umożliwia wahania skrętne. Na końcu osi
znajduje się śruba umożliwiająca mocowanie na niej brył.\\
\forceindent Wahadło skrętne jest szczególnym przypadkiem wahadła fizycznego. Jeżeli założymy, że wychylenia
wahadła są niewielkie (w naszym przypadku ok $90^{\circ}$) oraz zaniedbamy siły oporu, jego ruch można opisać jako ruch
harmoniczny prosty. W takim przypadku okres T drgań wahadła można zapisać następująco:
\begin{equation}
T = 2 \pi \sqrt{\frac{I}{D}},
\end{equation}
gdzie $I$ jest momentem bezwładności bryły zamocowanej na osi wahadła, a $D$ jest parametrem charakterystycznym dla danej sprężyny - jej momentem kierującym.

\forceindent Aby wyliczyć moment bezwładności bryły zamocowanej na wahadle zastosujemy przekształcone równanie (4), wynik pomiaru okresu drgań $T$ oraz wyznaczony moment kieryjący $D$:
\begin{equation}
I = D \left(\frac{T}{2\pi}\right)^2
\end{equation}

\forceindent W omawianym ćwiczeniu moment kierujący sprężyny wyznaczymy mierząc okresy wahań pręta obciążonego dwoma ciężarkami. Moment bezwładności układu złożonego z pręta z dwoma ciężarkami, umieszczonymi symetrycznie w odległości $r$ od osi obrotu, można zapisać następującym wzorem:
\begin{equation}
I = I_P + 2m_{C}r^2,
\end{equation}
gdzie $I_P$ jest momentem bezwładności pręta, a $m_C$ jest masą każdego ciężarka. \\
Na podstawie wzoru (5) moment bezwładności pręta zamocowanego na osi możemy zapisać równaniem:
\begin{equation}
I_P = D \left(\frac{T_P}{2\pi}\right)^2,
\end{equation}
gdzie $T_P$ jest okresem drgan wahadła obciążonego prętem (bez ciężarków).\\
Podstawiając do równania (5) wzory (6) i (7), otrzymujemy zależnośc:
\begin{equation}
D \left(\frac{T}{2\pi}\right)^2 = 2m_{C}r^2 + D \left(\frac{T_P}{2\pi}\right)^2,
\end{equation}
która po przekształceniu przybiera postać:
\begin{equation}
T^2 = \frac{8\pi^2m_C}{D}r^2 + T_{P}^2.
\end{equation}
Dokonując w powyższym równaniu podstawień, $y=T^2$, $x=r^2$, $a=8\pi^2m_C/D$ oraz $b=T_{P}^2$, uzyskuje się zależność typu $y=ax+b$. Jest to funkcja liniowa, gdzie wartość $a$ jest współczynnikiem kierunkowym prostej, a $b$ punktem przecięcia z osią $y$. Stąd też moment kierujący wahadła można obliczyć ze wzoru:
\begin{equation}
D=\frac{8\pi^2m_{C}}{a}.
\end{equation}
\forceindent W celu wyznacznia momentu bezwładności pręta można skorzystać z pomiaru okresu dla nieobciążonego pręta z równania (7). Chcąc wyznaczyć moment bezwładności dysku należy zamocować jesgo środek na osi, a następnie zmierzyć okres drgań wahadła. Korzystając z równania (5) wyznaczamy doświadczalną wartość momentu bezwładności dysku. Chcąc porównać uzyskane wartości doświadczalne z wartoścami teorytycznymi należy zważyć obie bryły, zmierzyć długość pręta oraz średnicę dysku. Wartości teorytyczne momentów bezwładności obliczymy z równań:
\begin{equation}
I = \frac{1}{12}ml^2,
\end{equation}
\begin{equation}
I = \frac{1}{2}MR^2,
\end{equation}
gdzie $m$ jest masą pręta, $I$ - długością pręta, $M$ - masą dysku, $R$ - promieniem dysku.\\
\forceindent Żeby potwierdzić twierdzenie Steinera, dysk będziemy przykręcać na osi dla różnych odległości $d$ od środka tarczy. Następnie dla każdego położenia wyznaczymy okres dragań wahadła a potem korzystając z równania (5), moment bezwładności dysku. Otrzymany wynik porównamy z teorytycznym momentem bezwładności, który obliczymy na podstawie równania:
\begin{equation}
I = \frac{1}{2}MR^2 + Md^2.
\end{equation} 
\newpage
\section{Wyniki pomiarów}
\forceindent Masa ciężarków, masa i długość pręta, odległości między nacięciami na pręcie zostały zmierzone na początku doświadczenia i wynosiły:\\
\begin{center}
\begin{tabular}{|c|c|c|}
\hline
Mierzona wartość & Pomiar & Dokładność pomiarowa\\
\hhline{|=|=|=|}
Masa ciężarków $m_C$ & 0.259 [kg] & $\pm$ 0.001 [kg]\\
Masa pręta & 0.136 [kg] & $\pm$ 0.001 [kg]\\
Masa dysku $M$& 0.435 [kg] & $\pm$ 0.001 [kg]\\
\hline
Długość pręta $l$& 0.62 [m] & $\pm$ 0.001 [m]\\
Średnica dysku 2$R$ & 0.32 [m] & $\pm$ 0.001 [m]\\
\hline
Odległość między nacięciami & 0.05 [m] & $\pm$ 0.001 [m]\\
\hline

\end{tabular}
\end{center}
\vspace{10pt}

\forceindent Dokładność pomiaru czasu ze względu na pomiar manualny wynosi $\pm$ 0.1s
\vspace{10pt}
\begin{table}[!h]
\centering
\begin{tabular}{|c||c|c|c|}
\hline
Odległość od osi obrotu [cm]& Próba 1 [s] & Próba 2 [s] & Próba 3 [s] \\
\hhline{|=||=|=|=|}
0 & 12.45 & 12.50 & 12.60 \\
5 & 14.21 & 14.16 & 14.37 \\ 
10 & 18.39 & 18.46 & 18.62 \\
15 & 24.04 & 24.09 & 24.00 \\
20 & 30.20 & 30.06 & 29.75 \\
25 & 36.29 & 36.47 & 36.53 \\
30 & 42.87 & 42.83 & 42.91 \\
\hline

\end{tabular}
\caption{Pomiary czasu $t$ 5 wahnięć pręta przy różnych odległościach ciężarków od osi obrotu}
\end{table}
\vspace{10pt}


\begin{table}[!h]
\centering
\begin{tabular}{|c||c|c|c|}
\hline
Odległość od osi obrotu [cm]& Próba 1 [s] & Próba 2 [s] & Próba 3 [s] \\
\hhline{|=||=|=|=|}
0 & 14.14 & 14.18 & 14.23 \\
2 & 14.26 & 14.22 & 14.18 \\ 
4 & 14.59 & 14.41 & 14.58 \\
6 & 15.77 & 15.86 & 15.98 \\
8 & 17.27 & 17.14 & 17.27 \\
10 & 19.08 & 18.97 & 18.99 \\
12 & 20.96 & 20.81 & 20.76 \\
14 & 22.80 & 22.84 & 22.83 \\
\hline

\end{tabular}
\caption{Pomiary czasu $t$ 5 wahnięć dysku przy różnych odległościach od osi obrotu}
\end{table}
\vspace{10pt}
\newpage
\section{Opracowanie wyników}
\subsection{Wyznaczenie momentu kierującego sprężyny}

\forceindent Dla każdego położenia ciężarków na pręcie wyznaczamy średni czas $t_{sr}$ 5 wahnięć, oraz średni okres jednego wahnięcia $T$, korzystając z zależności $T = \frac{1}{5}t_{sr}$. Wyniki przedstawiono w Tablicy 3.\\ 
\begin{table}[!h]
\centering
\begin{tabular}{|c||c|c|c|}
\hline

 & $r$ [m] & $t_{sr}$ [s] & $T$ [s] \\
\hline	
\hline
%Pręt & 12.517 & 2,503 \\ \hline
Ciężarki 5 cm od osi & 0.05 & 14.247 & 2,849\\ \hline
Ciężarki 10 cm od osi&  0.10 & 18.490 & 3,698 \\ \hline
Ciężarki 15 cm od osi&0.15 & 24.043 &4,809 \\ \hline
Ciężarki 20 cm od osi&0.20 & 30.003 &6,001 \\ \hline
Ciężarki 25 cm od osi&0.25 & 36.430 & 7,286 \\ \hline
Ciężarki 30 cm od osi&0.30 &  42.870 & 8,574 \\ \hline
\hline

\end{tabular}
\caption{}
\end{table}

\forceindent Następnie, korzystając z obliczonych wartości, na rysunku 1 przedstawiono wykres zależności kwadratu okresu od kwadratu odległości ciężarków od osi obrotu $T^2 = f(r^2)$.

\begin{figure}[!h]
\centering
\begin{tikzpicture}[scale=1]
\begin{axis}[
xlabel={Kwadrat odległości ciężarków od osi obrotu $r^2$ [$m^2$]},
ylabel={Kwadrat okresu $T^2$ [$s^2$]},
xmin=0,xmax=0.11,
ymin=0,ymax=80,
legend pos=north west,
ymajorgrids=true,grid style=dashed
]


\addplot[smooth, tension={0.3}, green, mark=*, mark size = {0.5pt}]
coordinates {
(0, 6.265009)
(0.0025, 8.116801)
(0.01, 13.675204)
(0.0225, 23.126481)
(0.04, 36.012001)
(0.0625, 53.085796)
(0.09, 73.513476)
};

\draw [draw=black] (0.5,79.17) rectangle ++(4, 4);
\draw [draw=black] (8,134.75) rectangle ++(4, 4);
\draw [draw=black] (20.5,229.26) rectangle ++(4, 4);
\draw [draw=black] (38,358.12) rectangle ++(4, 4);
\draw [draw=black] (60.5,528.86) rectangle ++(4, 4);
\draw [draw=black] (88,733.13) rectangle ++(4, 4);

\legend{$T^2 = f(r^2)$}
\end{axis}
\end{tikzpicture}
\caption{Zależności kwadratu okresu od kwadratu odległości ciężarków od osi obrotu}
\label{fig:wyk}
\end{figure}

\forceindent Posługując się metodą regresji liniowej opisanej za pomocą wzoru 
\begin{equation}
a=\frac{n\Sigma x_i y_i - \Sigma x_i \Sigma y_i}{n\Sigma x_i^2 - (\Sigma x_i)^2},
\end{equation}
gdzie $x_i$ to kolejne kwadraty odległości ciężarków od osi obrotu, a $y_i$ to kwadraty okresu dla danej odległości, wyznaczamy współczynnik nachylenia prostej $a$, oraz jego niepewność. 
\begin{equation}
a = 747.958 \left[\frac{s^2}{m^2}\right]
\end{equation}
Następnie korzystając z równania (10), wyliczamy moment kierujący $D$, oraz jego niepewność:
\begin{equation}
D=\frac{8\pi^2m_{C}}{a} \left[\frac{kg}{\frac{s^2}{m^2}}\right] = 0.02734087 \left[\frac{kg\cdot m^2}{s^2}\right]
\end{equation}

----------------------------------do tego momentu ready --------------------\\

Błąd wyznaczenia wielkości $a$:\\
$$ \Delta a = \sqrt{\frac{n(\Sigma y_i ^2 - a \Sigma x_i y_i - b\Sigma y_i)}{(n-2)(n \Sigma x_i ^2 - (\Sigma x_i)^2)}} = 2,852$$

Błąd wyznaczenia $D$ obliczamy stosując metodę różniczki logarytmicznej
$$ \Delta D = (\frac{\Delta m_c}{m_c} + \frac{\Delta a}{a}) * D = (\frac{0,001}{0,259} + \frac{2,852}{770,663})*0,026535 = 0,0002$$

\forceindent Zatem ostateczne wartości $a$ i $D$ wyglądają następująco:

\begin{table}[!h]
\centering
\begin{tabular}{|cc||c|}
\multicolumn{1}{c}{} & \multicolumn{1}{c}{$a$} & \multicolumn{1}{c}{$D$}\\
\hline
pomiar & 770,66250527943 & 0,0265353772617267\\
\hline
dokładność & 2,852 & 0,0002\\
\hline
po zaokrągleniu & $ 770,7 \pm 2,9 $  & $ 0,265 \pm 0,0002 $ \\
\hline
\end{tabular}
\caption{Współczynnik nachylenia linii $a$ i moment kierujący $D$ wraz z dokładnościami $\Delta a$ i $\Delta D$}
\end{table}

\begin{table}
\centering
\begin{tabular}{|c|c|c|c|c|c|}
\hline	
 & Śr. Czas pomiaru $t$ & Śr. Okres $T$ & Teoretyczny & Zmierzony & Niepewność\\
\hline
Pręt & 12.517 & 2,503 & 0,004357 & 0,004212 & 0,003000\\
\hline
Ciężarki 5 cm od osi & 14.247 & 2,849&0,005652&	0,005457&	0,006000\\
\hline
Ciężarki 10 cm od osi&  18.490 & 3,698 & 0,009537 &	0,009192&	0,006000\\
\hline
Ciężarki 15 cm od osi&24.043 &4,809 &0,016012&	0,015542&	0,006000\\
\hline
Ciężarki 20 cm od osi&30.003 &6,001 &0,025077&	0,024203&	0,006000\\
\hline
Ciężarki 25 cm od osi&36.430 & 7,286 &0,036732&	0,035682&	0,006000\\
\hline
Ciężarki 30 cm od osi& 42.870 & 8,574	&0,050977&	0,049412&	0,006000\\
\hline
\\
\hline
Dysk&14.185 & 2,837&0,005568&	0,005409&	0,003000\\
\hline
Dysk 2 cm od osi&14.22 & 2,844&0,005742&	0,005437&	0,006000\\
\hline
Dysk 4 cm od osi&14.525 & 2,905&0,006264&	0,005674&	0,006000\\
\hline
Dysk 6 cm od osi&15.87 & 3,174&0,007134&	0,006771&	0,006000\\
\hline
Dysk 8 cm od osi&17.225 & 3,445&0,008352&	0,007979&	0,006000\\
\hline
Dysk 10 cm od osi&19.015 & 3,803&0,009918&	0,009719&	0,006000\\
\hline
Dysk 12 cm od osi&20.845 & 4,169&0,011832&	0,011680&	0,006000\\
\hline
Dysk 14 cm od osi&22.825 & 4,565&0,014094&	0,014005&	0,006000\\
\hline
\end{tabular}
\caption{Tabela z wynikami}
\end{table}


\end{document}