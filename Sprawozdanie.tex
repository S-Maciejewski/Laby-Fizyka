\documentclass[10pt,a4paper]{article}

\usepackage{polski}
\usepackage[utf8]{inputenc}

\usepackage[polish]{babel}
\usepackage{hhline}
\usepackage{pgfplots}

\usepackage{geometry}
\geometry{a4paper, total={170mm,257mm}, left=20mm, top=20mm }

\author{Sebastian Maciejewski 132275 i Jan Techner 132332}
\title{Wyznaczanie zależności przewodnictwa od temperatury 
dla półprzewodników i przewodników - \\ doświadczenie 203 (sala 217A)}
\date{8 grudnia 2017}
\setlength{\parindent}{0pt}
\newcommand{\forceindent}{\leavevmode{\parindent=3em\indent}}
\begin{document}
\maketitle
\section{Wstęp teorytyczny}


\subsection*{Opis doświadczenia}

\section{Wyniki pomiarów}
\forceindent Dla ogrzewania i chłodzenia przewodnika i półprzewodnika otrzymaliśmy następujące odczyty oporu:\\
\begin{center}
\begin{tabular}{|c|c|c|}
\hline
Temperatura $(K)$ & Opór półprzewodnika $(k\Omega)$ & Opór przewodnika $(\Omega)$\\
\hline
295,95&208,0&109,1\\ 
 \hline 
299,45&177,0&110,4\\ 
 \hline 
304,45&144,0&112,1\\ 
 \hline 
309,45&117,0&114,1\\ 
 \hline 
314,45&95,0&116,0\\ 
 \hline 
319,45&80,1&117,9\\ 
 \hline 
324,45&66,5&119,8\\ 
 \hline 
329,45&54,7&121,7\\ 
 \hline 
334,45&46,4&123,5\\ 
 \hline 
339,45&39,2&125,3\\ 
 \hline 
344,45&33,0&127,1\\ 
 \hline 
349,45&28,0&129,1\\ 
 \hline 
354,45&23,9&130,8\\ 
 \hline 
359,45&20,4&132,6\\ 
 \hline 
354,45&26,8&131,3\\ 
 \hline 
349,45&31,6&129,9\\ 
 \hline 
344,45&38,1&128,1\\ 
 \hline 
339,45&44,9&126,4\\ 
 \hline 
334,45&52,7&124,6\\ 
 \hline 
329,45&61,2&122,9\\ 
 \hline 
324,45&73,4&120,7\\ 
 \hline 
319,45&87,0&118,6\\ 
 \hline 
314,45&103,6&116,6\\ 
 \hline 
309,45&124,7&114,6\\ 
 \hline 
304,45&149,9&112,5\\ 
 \hline 
299,45&181,6&110,5\\ 
 \hline 
298,05&191,7&109,9\\ 
 \hline 


\end{tabular}
\end{center}



\section{Opracowanie wyników}
\forceindent Dla zależności:
\begin{equation}
ln(1/R) = f(1/T)
\end{equation}  
wyliczymy teraz, korzystając z metody regresji liniowej, współczynnik nachylenia prostej.\\
Przyjmujemy, że $ln(1/R) = y$ i $1/T = x$.
Posługując się metodą regresji liniowej opisaną wzorem:
\begin{equation}
a=\frac{n\Sigma x_i y_i - \Sigma x_i \Sigma y_i}{n\Sigma x_i^2 - (\Sigma x_i)^2},
\end{equation}
wyznaczamy współczynnik nachylenia prostej $a$, oraz jego niepewność. 
\begin{equation}
a = -3869,397 \left[\frac{K}{\Omega}\right]
\end{equation}
Następnie korzystając z równania:
\begin{equation}
a = \frac{E_A}{2k} \Rightarrow E_A = 2ak
\end{equation}
obliczamy energię aktywacji ($E_A$), która wynosi:
$$ E_A = -1,068 * 10^-19 \frac{J}{K} = -0,667 \frac{eV}{K} $$

Błąd wyznaczenia wielkości $a$:\\
$$ \Delta a = \sqrt{\frac{n(\Sigma y_i ^2 - a \Sigma x_i y_i - b\Sigma y_i)}{(n-2)(n \Sigma x_i ^2 - (\Sigma x_i)^2)}} = $$


\forceindent Zatem ostateczne wartości $a$ i $E_A$ wyglądają następująco:

\begin{table}[!h]
\centering
\begin{tabular}{|cc||c|c|}
\multicolumn{1}{c}{} & \multicolumn{1}{c}{$a$} & \multicolumn{1}{c}{$E_A [\frac{J}{K}]$} & \multicolumn{1}{c}{$E_A [\frac{eV}{K}]$}\\
\hline
pomiar & $-3869,39702854943$ & $-1,068 * 10^-19$ & $-0,667$\\
\hline
dokładność & TODO & TODO & TODO\\
\hline
po zaokrągleniu & TODO  & TODO & TODO\\
\hline
\end{tabular}
\caption{Współczynnik nachylenia linii $a$ i energia aktywacji $E_A$ wraz z dokładnościami $\Delta a$ i $\Delta E_A$}
\end{table}




\end{document}