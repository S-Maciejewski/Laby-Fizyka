\documentclass[10pt,a4paper]{article}

\usepackage{polski}
\usepackage[utf8]{inputenc}

\usepackage[polish]{babel}
\usepackage{hhline}
\usepackage{pgfplots}

\usepackage{graphicx}
\usepackage{caption}
\usepackage{subcaption}

\usepackage{geometry}
\geometry{a4paper, total={170mm,257mm}, left=20mm, top=20mm }

\author{Sebastian Maciejewski 132275 i Jan Techner 132332}
\title{Wyznaczanie zależności przewodnictwa od temperatury 
dla półprzewodników i przewodników - \\ doświadczenie 203 (sala 217A)}
\date{8 grudnia 2017}
\setlength{\parindent}{0pt}
\newcommand{\forceindent}{\leavevmode{\parindent=3em\indent}}
\begin{document}
\maketitle
\section{Wstęp teorytyczny}
\forceindent Przewodnictwo właściwe materiałów zależy od temperatury. Dla metali (przewodników) spada przy wzroście temperatury ze względu na spadek ruchliwości nośników. W przypadku półprzewodnika samoistnego zdolność przewodzenia prądu rośnie wykładniczo przy wzroście temperatury. Dzieje się tak, gdyż rośnie koncentracja nośników. Ruchliwość spada podobnie jak w metalach, zmiany te są jednak niewielkie w porównaniu ze zmianami koncentracji.\\
Takie właściwości półprzewodnika wynikają z tego, iż nośnikami prądu są w nim elektrony w paśmie przewodnictwa i dziury w paśmie walencyjnym. Elektrony są dostarczane do pasma przewodnictwa z pasma walencyjnego (w półprzewodnikach samoistnych) lub z poziomów domieszkowych (w półprzewodnikach domieszkowanych). Dziury natomiast powstają w paśmie walencyjnym po przejściu elektronu do pasma przewodnictwa.\\
Liczba elektronów przechodzących na wyższy poziom energetyczny zależy wykładniczo min. od temperatury i wyraża się (dla półprzewodników samoistnych) wzorem:
\begin{equation}
n = n_{0_s} e^{\frac{E_g}{2kT}}
\end{equation}
gdzie $E_g$ to szerokość pasma zabronionego, $k$ to stała Boltzmana a $T$ temperatura.

\subsection*{Opis doświadczenia}



\newpage
\section{Wyniki pomiarów}

------------------ DOKŁADNOŚĆ POMIARÓW ------------------------ \\

\forceindent Dla ogrzewania i chłodzenia przewodnika i półprzewodnika otrzymaliśmy następujące odczyty oporu:\\
%\subsection*{Ogrzewanie}
\vspace{10pt}
\begin{center}
\begin{tabular}{|c|c|c|}
\hline
Temperatura $[K]$ & Opór półprzewodnika $[k\Omega]$ & Opór przewodnika $[\Omega]$\\
\hline
295,95&208,0&109,1\\ 
 \hline 
299,45&177,0&110,4\\ 
 \hline 
304,45&144,0&112,1\\ 
 \hline 
309,45&117,0&114,1\\ 
 \hline 
314,45&95,0&116,0\\ 
 \hline 
319,45&80,1&117,9\\ 
 \hline 
324,45&66,5&119,8\\ 
 \hline 
329,45&54,7&121,7\\ 
 \hline 
334,45&46,4&123,5\\ 
 \hline 
339,45&39,2&125,3\\ 
 \hline 
344,45&33,0&127,1\\ 
 \hline 
349,45&28,0&129,1\\ 
 \hline 
354,45&23,9&130,8\\ 
 \hline 
359,45&20,4&132,6\\ 
 \hline 
\end{tabular}
\end{center}

%\subsection*{Ochładzanie}
%\vspace{10pt}
%\begin{center}
%\begin{tabular}{|c|c|c|}
%\hline
%Temperatura $[K]$ & Opór półprzewodnika $[k\Omega]$ & Opór przewodnika $[\Omega]$\\
%\hline
%359,45&20,4&132,6\\ 
% \hline 
%354,45&26,8&131,3\\ 
% \hline 
%349,45&31,6&129,9\\ 
% \hline 
%344,45&38,1&128,1\\ 
% \hline 
%339,45&44,9&126,4\\ 
% \hline 
%334,45&52,7&124,6\\ 
% \hline 
%329,45&61,2&122,9\\ 
% \hline 
%324,45&73,4&120,7\\ 
% \hline 
%319,45&87,0&118,6\\ 
% \hline 
%314,45&103,6&116,6\\ 
% \hline 
%309,45&124,7&114,6\\ 
% \hline 
%304,45&149,9&112,5\\ 
% \hline 
%299,45&181,6&110,5\\ 
% \hline 
%298,05&191,7&109,9\\ 
% \hline 
%\end{tabular}
%\end{center}


\newpage
\section{Opracowanie wyników}

\forceindent Poniższe wykresy przedstawiają zależności zmierzonych wartości oporu przewodnika i połprzewodnika od temperatury:  

\begin{figure}[!h]
\centering
\begin{minipage}{.5\textwidth}
\centering
\begin{tikzpicture}[scale=0.9]
\begin{axis}[
xlabel={Temperatura [$K$]},
ylabel={Opór przewodnika [$\Omega$]},
xmin=290,xmax=365,
ymin=107,ymax=135,
legend pos=north west,
ymajorgrids=true,grid style=dashed]

\addplot[smooth, tension={0.3}, red, mark=*, mark size = {0.5pt}]
coordinates {
(295.95, 109.1)
(299.45, 110.4)
(304.45, 112.1)
(309.45, 114.1)
(314.45, 116.0)
(319.45, 117.9)
(324.45, 119.8)
(329.45, 121.7)
(334.45, 123.5)
(339.45, 125.3)
(344.45, 127.1)
(349.45, 129.1)
(354.45, 130.8)
(359.45, 132.6)
};
%\draw [draw=black] (0.5,79.17) rectangle ++(4, 4);
%\draw [draw=black] (8,134.75) rectangle ++(4, 4);
%\draw [draw=black] (20.5,229.26) rectangle ++(4, 4);
%\draw [draw=black] (38,358.12) rectangle ++(4, 4);
%\draw [draw=black] (60.5,528.86) rectangle ++(4, 4);
%\draw [draw=black] (88,733.13) rectangle ++(4, 4);
\legend{$R = f(T)$}
\end{axis}
\end{tikzpicture}
\end{minipage}%
\begin{minipage}{.5\textwidth}
\centering
\begin{tikzpicture}[scale=0.9]
\begin{axis}[
xlabel={Temperatura [$K$]},
ylabel={Opór półprzewodnika [$\Omega$]},
xmin=290,xmax=365,
ymin=0,ymax=230000,
legend pos=north east, ymajorgrids=true,grid style=dashed]

\addplot[smooth, tension={0.3}, green, mark=*, mark size = {0.5pt}]
coordinates {
(295.95, 208000.0)
(299.45, 177000.0)
(304.45, 144000.0)
(309.45, 117000.0)
(314.45, 95000.0)
(319.45, 80000.1)
(324.45, 66000.5)
(329.45, 54000.7)
(334.45, 46000.4)
(339.45, 39000.2)
(344.45, 33000.0)
(349.45, 28000.0)
(354.45, 23000.9)
(359.45, 20000.4)
};
%\draw [draw=black] (0.5,79.17) rectangle ++(4, 4);
%\draw [draw=black] (8,134.75) rectangle ++(4, 4);
%\draw [draw=black] (20.5,229.26) rectangle ++(4, 4);
%\draw [draw=black] (38,358.12) rectangle ++(4, 4);
%\draw [draw=black] (60.5,528.86) rectangle ++(4, 4);
%\draw [draw=black] (88,733.13) rectangle ++(4, 4);
\legend{$R = f(T)$}
\end{axis}
\end{tikzpicture}
\end{minipage}
\caption{Zależność oporu przewodnika i półprzewodnika od temperatury}
\end{figure}

\forceindent W tabeli poniżej znajdują się wartości zmierzonej temperatury (w kelwinach), jej odwrotności oraz logarytm naturalny odwrotności oporu półprzewodnika w danej temperaturze: 

\begin{center}
\begin{tabular}{|c|c|c|}
\multicolumn{1}{c}{$T (K)$} & \multicolumn{1}{c}{$1/T$} & \multicolumn{1}{c}{$ln(1/R)$}\\
\hline
295,95&0,00337895&-12,24529336\\ 
 \hline 
299,45&0,00333946&-12,08390501\\ 
 \hline 
304,45&0,00328461&-11,87756858\\ 
 \hline 
309,45&0,00323154&-11,66992921\\ 
 \hline 
314,45&0,00318016&-11,46163217\\ 
 \hline 
319,45&0,00313038&-11,29103113\\ 
 \hline 
324,45&0,00308214&-11,10495723\\ 
 \hline 
329,45&0,00303536&-10,90961899\\ 
 \hline 
334,45&0,00298998&-10,74505474\\ 
 \hline 
339,45&0,00294594&-10,57643203\\ 
 \hline 
344,45&0,00290318&-10,40426284\\ 
 \hline 
349,45&0,00286164&-10,23995979\\ 
 \hline 
354,45&0,00282127&-10,08163374\\ 
 \hline 
359,45&0,00278203&-9,92329018\\ 
% \hline 
%354,45&0,00282127&-10,19615717\\ 
% \hline 
%349,45&0,00286164&-10,3609124\\ 
% \hline 
%344,45&0,00290318&-10,54796956\\ 
% \hline 
%339,45&0,00294594&-10,71219307\\ 
% \hline 
%334,45&0,00298998&-10,87237073\\ 
% \hline 
%329,45&0,00303536&-11,02190247\\ 
% \hline 
%324,45&0,00308214&-11,20367921\\ 
% \hline 
%319,45&0,00313038&-11,3736634\\ 
% \hline 
%314,45&0,00318016&-11,54829261\\ 
% \hline 
%309,45&0,00323154&-11,73366613\\ 
% \hline 
%304,45&0,00328461&-11,91772368\\ 
% \hline 
%299,45&0,00333946&-12,10956175\\ 
% \hline 
%298,05&0,00335514&-12,16368693\\ 
 \hline 

\end{tabular}
\end{center}

\forceindent Wartości z tabeli przedstawione na wykresie:

\begin{figure}[!h]
\centering
\resizebox{!}{6cm}{
\begin{tikzpicture}[scale=1]
\begin{axis}[
xlabel={$\frac{1}{T}$},
ylabel={$ln(\frac{1}{R})$},
xmin=0.0027,xmax=0.0035,
ymin=-12.5,ymax=-9.5,
legend pos=north east,
ymajorgrids=true,grid style=dashed
]


\addplot[smooth, tension={0.3}, cyan, mark=*, mark size = {0.5pt}]
coordinates {
(0.00337895, -12.24529336)
(0.00333946, -12.08390501)
(0.00328461, -11.87756858)
(0.00323154, -11.66992921)
(0.00318016, -11.46163217)
(0.00313038, -11.29103113)
(0.00308214, -11.10495723)
(0.00303536, -10.90961899)
(0.00298998, -10.74505474)
(0.00294594, -10.57643203)
(0.00290318, -10.40426284)
(0.00286164, -10.23995979)
(0.00282127, -10.08163374)
(0.00278203, -9.92329018)
};

%\draw [draw=black] (0.5,79.17) rectangle ++(4, 4);
%\draw [draw=black] (8,134.75) rectangle ++(4, 4);
%\draw [draw=black] (20.5,229.26) rectangle ++(4, 4);
%\draw [draw=black] (38,358.12) rectangle ++(4, 4);
%\draw [draw=black] (60.5,528.86) rectangle ++(4, 4);
%\draw [draw=black] (88,733.13) rectangle ++(4, 4);

\legend{$ln(\frac{1}{R}) = f(\frac{1}{T})$}
\end{axis}
\end{tikzpicture}
}
\caption{Zależność logarytmu naturalnego odwrotności oporu półprzewodnika od odwrotności temperatury}
\label{fig:wyk}
\end{figure}
\vspace{10pt}

\forceindent Dla zależności:
\begin{equation}
ln(1/R) = f(1/T)
\end{equation}  
wyliczymy teraz, korzystając z metody regresji liniowej, współczynnik nachylenia prostej.\\
Przyjmujemy, że $ln(1/R) = y$ i $1/T = x$.
Posługując się poniższymi wzorami:
\begin{equation}
a=\frac{n\Sigma x_i y_i - \Sigma x_i \Sigma y_i}{n\Sigma x_i^2 - (\Sigma x_i)^2},
\end{equation}
\begin{equation}
b=\frac{1}{n} \left( \Sigma y_i - a\Sigma x_i \right)
\end{equation}
wyznaczamy współczynnik $a$ oraz punkt $b$ przecięcia prostej z osią OY:
\begin{equation}
a = -3869,397 \left[\frac{K}{\Omega}\right]
\end{equation}
\begin{equation}
b = 0.0004225 
\end{equation}
Błąd wyznaczenia wielkości $a$:\\
$$ \Delta a = \sqrt{\frac{n(\Sigma y_i ^2 - a \Sigma x_i y_i - b\Sigma y_i)}{(n-2)(n \Sigma x_i ^2 - (\Sigma x_i)^2)}} = 2.60816$$
Następnie korzystając z równania:
\begin{equation}
a = \frac{E_A}{2k} \Rightarrow E_A = 2ak
\end{equation}
obliczamy energię aktywacji $E_A$, która wynosi:
$$ E_A = -1,068 * 10^-19 \frac{J}{K} = -0,667 \frac{eV}{K} $$





\forceindent Zatem ostateczne wartości $a$ i $E_A$ wyglądają następująco:

\begin{table}[!h]
\centering
\begin{tabular}{|cc||c|c|}
\multicolumn{1}{c}{} & \multicolumn{1}{c}{$a$} & \multicolumn{1}{c}{$E_A [\frac{J}{K}]$} & \multicolumn{1}{c}{$E_A [\frac{eV}{K}]$}\\
\hline
pomiar & $-3869,39702854943$ & $-1,068 * 10^-19$ & $-0,667$\\
\hline
dokładność & TODO & TODO & TODO\\
\hline
po zaokrągleniu & TODO  & TODO & TODO\\
\hline
\end{tabular}
\caption{Współczynnik nachylenia linii $a$ i energia aktywacji $E_A$ wraz z dokładnościami $\Delta a$ i $\Delta E_A$}
\end{table}




\end{document}
